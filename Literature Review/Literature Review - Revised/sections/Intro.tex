\documentclass[../sample.tex]{subfiles}

\begin{document}

This review attempts to present the current state-of-the-art in regard to solving the ‘Multi-Agent
Problem’, which considers a scenario of multiple agents (such as robots, AI systems etc) must
interact, whether cooperatively or competitively, to achieve defined goals. To this end, this review
categorises work based on the approach towards solving this problem. Each approach often places
assumptions on the type of agents present within the system and so each is appropriate in different
settings. Briefly, these categories are

\begin{itemize} 
\item Swarms, in which the agents are often assumed to be homogeneous and with limited sensing and
communication capabilities. 

\item Decentralised-Partially Observable Markov Decision Processes (Dec-POMDPs), in which agents
must choose actions with the aim to optimise a known loss function which is applied to the entire
team. 

\item Game Theory, which allows for each agent to have an individual payoff function which they
must optimise with respect to the actions of the other agents.  

\item Multi-Agent Reinforcement Learning, in which agents do not immediately have access to the
payoff function and so must determine it through iterations of exploration. 
	
\item Control Theory, which considers the low-level operation of each agent and mathematically
defines control laws for each agent which satisfy properties such as stability and controllability.

	\item Hard Coded, a term of the author’s devising to categorise systems which do not fit in any of
the formal methods above. These systems are built on a series of if-then statements, allowing agents
to reason about their current state and future tasks. 

\end{itemize}

\section{Objective}

The aim of the following sections is not to provide an exhaustive list of all work done in the
aforementioned areas. To attempt to do this would be an exercise in futility. Instead, it is to
identify research directions which lie within the broad scope of Multi-Agent Systems (MAS).  Once
identified, these directions will form the basis for the remainder of this review, allowing for
particular problems to emerge. It will likely be the case that a medley of these problems will be
addressed throughout the course of the PhD and, of course, more will likely be added. 


\end{document}