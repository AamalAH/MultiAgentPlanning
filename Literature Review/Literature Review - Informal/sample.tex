\documentclass[preprint,11pt]{report}

\usepackage{matlab-prettifier}
\usepackage[english]{babel}
\usepackage[utf8]{inputenc}
\usepackage{amsmath}
\usepackage{float}
\usepackage{graphicx}
\usepackage{caption}
\usepackage{subcaption}
\usepackage{csquotes}
\usepackage{algpseudocode}
\usepackage{algorithm}
\usepackage[colorinlistoftodos]{todonotes}
\usepackage[margin=0.5in,letterpaper]{geometry}
\def\changemargin#1#2{\list{}{\rightmargin#2\leftmargin#1}\item[]}
\let\endchangemargin=\endlist 
\usepackage{listings}
\def\changemargin#1{\list{}{\leftmargin#1}\item[]}
\let\endchangemargin=\endlist
\title{Literature Review - Informal}

\author{Aamal Hussain}

\date{\today}

\begin{document}

\maketitle 

\tableofcontents

\chapter*{Introduction}



The areas of interest regarding Multi Agent Systems fall into two distinct categories (which have
some slight overlap). The first is denoted as 'Decision Making', though the terms 'Planning' and
'Distributed Control' may sometimes be used as substitutes. It is important to note that, in my
case, I do not include Planning Formalisation techniques such as PDDL. Instead, I focus on the
particular methods which involve interaction in the world. The second, I will refer to as 'Multi
Agent Dynamics', though sometimes I will refer to this as 'Stability Analysis'. 

\subsection*{Scope}

Both of the aforementioned topics have been applied to all variants of multi agent systems and have
sufficient room for further exploration. The typical variants of multi agent systems that I will
consider are

\begin{itemize} \item Swarms \item Multi Agent Reinforcement Learning (MARL) \item Decentralised
Partially Observed Markov Decision Processes (Dec-POMDP) \item Game Theoretic Approaches \item Hard
Coded \end{itemize}

In the above, 'Game Theoretic Approaches' cover a wide spectrum, including zero-sum (minimax) games,
bayesian games, and many more. The final category, 'Hard Coded', refers to any approach which does
not fit neatly into any of these categories. The name is chosen since they often refer to methods
which revolve around a series of if-else statements.

I will first consider Decision Making as this is most closely related to the research proposal,
before I consider Multi Agent Dynamics, which has strong implications for Safe and Trusted Multi
Agent Systems. 

\subsection*{Objectives}

The aim of the following sections is not to provide an exhaustive list of all work done in the
aforementioned areas. To attempt to do this would be an exercise in futility. Instead, it is to
identify research directions which lie within the broad scope of Multi-Agent Systems (MAS). With
these directions, we may be able to narrow down the remaining literature review and hone in on
particular problems and, in fact, we may find that we address multiple of these over the next four
years (while we, of course add more). As such, the end of each chapter will provide a list of
research directions which I have identified from the preceding review. 

\chapter{Decision Making}

Decision making refers to the generalised problem of considering how a multi-agent system should
interact in the world to achieve their goals. This subsumes both the cases where the agents' goals
are aligned (cooperative) or in conflict with one another (competitive). 

The following sections will provide an overview of the literature aimed towards coordinating such
systems as followed by interesting sub-fields in each approach which present avenues for research
related to Safe and Trusted AI (STAI). These, of course, are not exhaustive and more will be added
as the review progresses.

\section{Swarms}

Swarm based systems comprise of multiple homogeneous agents which are able to organise themselves
through a formation using a series of simple local interactions with their neighbours
\cite{Couceiro2015}. Whilst the individual agents are generally simplistic, the collective behaviour
may exhibit complex phenomena emulating systems observed in biological organisations such as bee or
ant colonies \cite{Sethi2017}. Hybrid algorithms such as in \cite{Gao2018} show an accelerated
performance in reaching globally optimal solutions in search-based tasks. The advantage of many
swarm algorithms is that they are based on local interactions and so are incredibly scalable
\cite{Rizk2018}. 

The reduction of the complexity in interactions between agents also allows for the robots to perform
other calculations on board. In \cite{Pini2011TaskSelection}, Pini et al. leverage this by
considering adaptive task partitioning across swarms. This allows a swarm, in a decentralised
manner, to deliberate whether to partition a task into its sub-tasks or to perform the task in its
whole. As of now (to the best of my knowledge) the problem of partitioning general tasks into its N
sub-tasks is unexplored. This, however, highlights another advantage of swarm systems; they are
readily divided into sub-groups (as in \cite{Zahadat2016DivisionInhibition}) to perform a
divide-et-impera (divide and conquer) approach to solving problems \cite{Pini2011TaskSelection}). 

Furthermore, swarm systems may be designed in a leaderless manner and so do not require the use of a
central controller \cite{Couceiro2015}. This presents the advantage that the system can rapidly
adapt to the loss of agents or separation of groups throughout the task. However, the assumptions
made regarding the homogeneity of individual agents and the simplicity of their local interactions
result in significant limitations placed on the complexity of the tasks that swarm systems can
accomplish. 

\subsection{Co-evolution and Self-Healing}

Recently, there has been an increased interest in introducing heterogeneity into the swarm systems
to improve their real world applicability. An example of this which have been shown strong real
world success can be found in \cite{DucatelleSelf-organizedSwarms}. Here, the authors consider two
swarm teams, referred to as 'foot bots' and 'eye bots' who work in unison to explore an environment
and solve a navigation task. 

This area of research is sparsely populated and warrants further exploration. This is since the use
of co-evolutionary teams can improve the robustness of the swarm optimisation. This is since it will
be possible for a team to automatically determine when robots in the other team are not exhibiting
expected behaviour and ensure that the other team self corrects. This process is referred to as
'Self-Healing' in \cite{LiuTrust-Aware}. Here, the authors allow a user to define the goals of a
swarm system. From this, a 'trust' metric can be defined which measures the deviation of each agent
from the expected behaviour. The self-healing process occurs by limiting communication of all agents
with 'untrusted' agents and encouraging communication with 'trusted' agents. The unison of
self-healing with co-evolution may present the opportunity for heterogenous swarms to maintain their
evolutionary stability, even in the face of environmental disturbances.

\subsection{Fault Detection}

The above sections have considered the fact that swarm systems are robust to losses in the group.
However, any MAS system must first be able to recognise that an agent has undergone some failure. 

To this end, Tarapore et al. \cite{Tarapore2019FaultDetection} develop a robust fault detection
approach in which the swarm itself, in a decentralised manner, is capable of assessing deviation
from 'normal' behaviour, even when the behaviour of the swarm itself is altered (perhaps by a remote
operator). The authors achieve this by requiring that the agents themselves sense and characterise
their own behaviour. This characterisation is formulated as a binary feature vector which is then
communicated to the agent's neighbours. These neighbours will reach a consensus over whether the
agent should be treated as faulty based on their collective behaviour. The results presented in
\cite{Tarapore2019FaultDetection} show extremely promising results and suggest that their method is,
in fact, able to determine faults with high accuracy in the presence of various fault types
(including sensor and actuator failures), although poor performance is seen in actuation failures in
some instances. It should be noted that this method requires that each robot transmit their feature
vector to the nearest neighbours. In environments where communication may be severely limited, this
may present further errors. Furthermore, it is unlikely that, when a robot is damaged, only one of
its components will be affected. Therefore, it is important to determine the effect on performance
in the face of multiple agent failures and in communication losses. This exploration may open the
possibility of improving the state of the art in terms of failure detection in swarm systems. (Of
course, this conclusion is based off two papers so further review is required).

\subsection{Verification}

Both of Alessio's papers \cite{Kouvaros2019FormalSystems, Lomuscio2019ASystems} fit in here but they
require further reading

\subsection{Directions}

From the above consideration of swarms, the following research directions have been identified
which, in my view, concern themselves with STAI.

\begin{itemize} \item Healing through co-evolution: The application of heterogenous robot swarms
towards ensuring that emergent phenomenon and swarm behaviour are as expected by the user. \item
Fault Detection in limited communication: Considering the ability of a swarm to, in a decentralised
approach, consider which robots in the team have failed, even in cases of no communication or
multiple failures. \end{itemize}

\section{Dec-POMDPs}

The use of POMDPs in multi-agent settings is formalised as decentralised POMDP (Dec-POMDP) which
aims for a team of agents to maximise a common utility. However, it has been found that determining
the exact solution to Dec-POMDP problems is NEXP \cite{Eker2011} and so is intractable for all but
toy problems. A number of methods have been presented to attempt to solve Dec-POMDPs. Oliehoek gives
a review of these in \cite{OliehoekDecentralizedPOMDPs}. Most solutions (such as brute force) are
intractable for all but toy problems. 

Approximate solutions to Dec-POMDP have been proposed, perhaps most notable of which is the proposal
of MacDec-POMDP \cite{Amato2015} by Amato et al. Here, macro actions (actions which extend over
multiple time steps) are used, as opposed to low-level actions which are re-evaluated at each time
step. This allows an exact solution to be found as it does not need to be evaluated at each time
step. This method assumes that, once macro-actions are distributed, the policies (sequence of
state-action pairs) are known. Since this is not the case, Amato also proposes the use of a
Dec-POSMDP \cite{Amato2017Decision-MakingLearning}, where 'SMDP' refers to 'Semi-Markov Decision
Process, in which a high level model is defined without the underlying Dec-POMDP's actions and
observations.

However, this is largely applicable in passive settings where common payoffs can be determined by an
offline planner. They also require a significant amount of data with which to allow the system to
learn the underlying models and payoff structures. This limits the applicability of the system when
communication is limited and the system is presented with environments that it has not seen before.
Recent work in MDPs \cite{Klima2019RobustDomains} has considered learning in the face of Significant
Rare Events (SREs) which the system has not yet observed. Currently, it is required that a model of
such SREs is known and so it would be interesting to consider the application of Dec-POMDPs in
situations where the SRE model is incomplete or erroneous and assess the robustness of the Dec-POMDP
framework against such events. 

\subsection{Directions}

The area of Dec-POMDPs still requires more review from me, especially for solutions which are not
developed by Amato as he seems to largely dominate the field. Based on this initial review of the
area, a potential direction for research is

\begin{itemize} \item Significant Rare Events: Consider Dec-POMDP capability to remain robust to
SREs which have a limited or erroneous model. \item Agent Failures: The Dec-POMDP model uses a
centralised planner which acts offline. It therefore assumes that the agents will be able to carry
out their assigned tasks. It would be interesting to examine the possibility of, either adaptive
planning, or planning with contingencies in the Dec-POMDP framework. \end{itemize}

\section{Game Theoretic Approaches}

Game theoretic models are generally the go-to method for understanding multi-agent systems. As such
they fit into all of the categories in this chapter (except, perhaps, swarms) since Dec-POMDP and
MARL methods have both used game theory to support their frameworks. In fact, Dec-POMDP is a subset
of Partially Observed Stochastic Games (POSG), in which all agents use the same payoff. Game theory
can, therefore, be used in an applied capacity to direct task allocation across heterogenous teams.

\subsection{Market Based Methods}

Garapati et al. \cite{Garapati2018AMissions} define a market based method as the setting where
agents "follow their own interests and establish the mechanism of a market for distributing the
tasks". Auctioning is the most widely used sub-field of market approaches and so I will use them
interchangeably. 

Whilst there are different variants to autioning, the general procedure is that an auctioneer who
has knowledge of a task (or multiple tasks) will set up an auction for said task. Agents can then
make bids on these tasks and, once the auction is complete, the highest bid will win the task. In
the specific application to robotics, a robot's bid will often reflect the costs, suitability or
utility their undertaking the task \cite{BernardineDias2006Market-basedAnalysis}. This immediately
highlights a few points. The first is that the method is not too heavily reliant upon a single
processor to determine some joint policy. Tasks are allocated on a case-by-case basis and the
utilities are calculated by the agent themselves. The only centralised process is the auctioneer's
assessment of the winner which then relays this information back to them. The downside of this is
that the system is heavily reliant upon strong communication channels, without which tasks may not
be assigned, incorrect utilities may be communicated and, in general, sub-optimal solutions reached.
Furthermore, the requirement that the agents themselves determine the cost of their actions assumes
that they have the computational capability to do so. Furthermore, the bids placed by each agent
need to be a strong representation of their capability to perform a task which may be hard to
estimate without expert knowledge. However, market based methods are well suited to explanation
through argumentation (similar to \cite{Jung2001DistributedArgumentation}. 

With well chosen payoffs, market based approaches work extremely well. For instance, in
\cite{Dias2000ASystem}, the authors show that a free market approach (where agents try to maximise
their own profits) can lead to a strong collaborative effort across teams. Similarly, in
\cite{Thomas2005Multi-robotScenarios}, Thomas et al. apply the auctioning scheme presented in
\cite{Gerkey2002Sold:Coordination} towards a robot construction team. However, it is important to
note that these are both passive settings; tasks were assigned before the team were in the field
and, in the case of \cite{Gerkey2002Sold:Coordination}, the system would repeat the bidding process
if a robot failed. While both show strong performance, it cannot be said that either would be
applicable in dangerous environments in which dynamic reassignment must happen within strict time
constraints. Stancliff et al. \cite{Stancliff2009PlanningAllocation} suggest that a more robust
method to planning would be to account for failures a priori, a philosophy which is exemplified in
\cite{Chen2010ACollaboration} who consider the robot's reliability and relevance to a task as well
as 'history relevance' which considers the relationship between pairs of robots with the aim of
producing more effective teams.

There has also been some interesting work in probabilistic verification of market based approaches.
Most notable to me is \cite{Pallottino2007ProbabilisticAvoidance} which considers the case of
conflict avoidance. Though their method focuses on collision avoidance, it highlights the need for
verification of conflict resolution and goal achievement in market based approaches with different
payoff structures. Sirigineedi et al. \cite{Sirigineedi2010DecentralisedApproach} make a step in
this direction by considering the verification of cooperative surveillance along a route network.
From my understanding, this means that they were able to verify that their agents were able to
traverse along the network without interference. However, this, as always, requires further analysis
to truly understand.

\subsection{Directions}

The particular considerations which have jumped out to me from the above analysis are as follows:

\begin{itemize} 
	\item Verification of goal achievement under different payoffs: can we ensure that
	self-interested agents will, in fact, show cooperative emergent properties? A similar question
	arises in terms of valuations (how much cost each agent incurs). 
	\item A priori consideration of
	reliability: Can we learn and take into account the fact that robots may fail throughout the
	progress of a mission when we assign tasks? 
\end{itemize}

\subsection*{Stochastic Games}

\section{Hard Coded}

\section{MARL}

Reinforcement learning extends the Markov Decision Process problem by considering the case where the
payoff model is not known. This, of course, is the case for most real world environments. As such,
MARL algorithms can perhaps be considered to be more applicable than Dec-POMDP models. Fortunately,
MARL has picked up a lot of traction in research recently, with a large body dedicated towards
solving the many problems it presents.

The largest problem in MARL is the non-stationarity of the environment \cite{Hernandez-LealA}. In
single-agent settings, it is assumed that the environment is Markovian. However, this must be lifted
in the Multi Agent setting since other agents in the environment will be learning concurrently. This
learning will be based on their own history of interactions which extend beyond the previous state.
As such, we must now consider that the policy for any one agent will depend on the policy of all
other agents. As such, a big concern in this area is regarding convergence guarantees and the
stability of the learner system. Approaches to this will be discussed in the next chapter. 

\subsection*{Agent Modelling}

Returning to the problem of non-stationarity, solutions have been presented in which the agent
models the learning of other agents. A noteworthy example of this is found in
\cite{Foerster2018LearningAwareness} in which the agent performs a one-step lookahead of the other
agents' learning and optimises with respect to this expected return. They show that this leads to
stable learning and can even lead to emergent cooperation from competition. However, the method
requires that both agents have exact knowledge of the others' value functions in order to perform
the one step lookahead. Furthermore, it has only been considered for the case of a two agent
adversarial game and so the scalability of the system to multiple agents is not yet understood.
Another method presented by Mao et al. \cite{MaoModellingDDPG} uses a centralised critic to collect
the actions and observations of all agents and allows it to model the joint policy of teammates.
This is shown to generate cooperative behaviour across four agents and so is more applicable to real
world settings. However, its disadvantage over the method presented in
\cite{Foerster2018LearningAwareness} is that the critic is centralised. In real world settings, this
requires the presence of an agent (perhaps a laptop) which is able to handle the computational load
of determining a joint policy across all agents and must then communicate the Q-values of all agents
back to them. This is both a taxing both in terms of computation and time. 

Hong et al \cite{Hong2018ASystems} present a similar system for modelling teammate policies by
tasking a CNN with determining the policy features of other agents and then embedding these as
features in its own DQN. This shows strong performance in settings where other agents dynamically
change their policies. The concerns with this, however, are that, as the number of agents in the
field increase, the CNN in each agent must perform another approximation. This places strong
requirements on the performance of the CNN since errors in estimation will accumulate as the number
of agents increases. Similarly, the complexity of the DQN will increase as more feature vectors are
added. 

Finally, all of the above methods are not robust to evolving numbers of agents. The problem of agent
modelling is an important one to ensure stable learning and to understand the evolution of the
system. It also presents a strong challenge and is open to exploration. To put it in context the
methods described in this section are all from 2018-19, so its all very new.

\subsection{Directions}

I still have a lot of reading to do regarding MARL, which, in turn, will identify new directions.
However, on initial assessment I put forward

\begin{itemize} \item Modelling evolving teammates: The purpose of this is to more strictly ensure
the stability of the learning process. However, the particular problem I suggest is to consider the
modelling in a decentralised manner and with the consideration of evolving numbers of agents in
teams. \end{itemize}


\section{MARL Approaches} \label{sec::MARL Approaches}

The task of MARL is to determine an optimal joint policy for all agents across the game. This joint
policy may be the concatenation of all the individual policy or it may just be options for each
agent to take. In either case, optimality is defined through the standard notions of Nash equilibria
and so, in this section, I will try to consider the broad spectrum of methods which attempt to
achieve this Nash equilibria. I will start with the more foundational methods which are, thankfully,
all reviewed by Schwartz in  \cite{SchwartzMulti-agentApproach}. I will use the chapters of the book
to help guide this review. I am not referencing the individual papers immediately, though the
references can be found in the book.

\subsubsection*{Learning in Two Player Matrix Games}

The most fundamental method to learning in Matrix games is the simplex algorithm. This is a popular
method of linear programming (in which constraints are linear). This will be important in
considering more current methods. A similar consideration is given to the infinitesimal gradient
ascent algorithm, in which the step size converges to zero. This method guarantees that, in the
infite horizon limit, the payoffs will converge to the Nash equilibrium payoff. Note that this does
not necessarily mean that both agents will converge to a single Nash equilibrium. This is a
particular problem in games where there are multiple Nash equilibria. However, in practice it is
difficult to choose a convergence rate of the step size and, without an appropriate choice the
strategy may oscillate as shown in the book. To address this, a modified approach is presented by
Bowling and Veloso which incorporates the notion of Win or Learn Fast (WoLF) to produce WoLF-IGA.
WoLF is a notion we will come across often in MARL and is shown by the authors to converge to always
to a NE. The concern with WoLF methods, however, is that it requires explicit knowledge of the
payoff matrix (which is not so much of a problem for model based methods) and the opponent's
strategy (which is more of a problem in real-world methods). Finally, the Policy Hill Climbing
method (PHC) is shown to converge to an optimal mixed strategy if the other agents are stationary
(i.e. are not learning). However, it is shown that, when this is not the case, the algorithm again
oscillates. The WoLF-PHC adaptation of this method is shown to converge to a NE strategy for both
players with minimal oscillation. 

The above methods are all centralised techniques, in which a controller determines the optimal joint
policy for both agents. However, in real scenarios it is often preferable that each agent learns
their own strategy, a task which must be completed without information of the other agent's
strategy. The methods presented towards this problem are: linear reward-inaction ($L_{R-I}$) which
guarantees convergence to NEs in games which contain pure NEs, linear reward-penalty ($L_{R-P}$)
which can guarantee convergence to mixed strategies given the appropriate parameters, lagging anchor
algorithm which also converges to mixed strategies, and the author's own proposal of the $L_{R-I}$
lagging anchor algorithm which can converge to both pure and mixed NEs.

\subsubsection*{Learning in Multiplayer Stochastic Games}

Stochastic Games (or Markov Games) form a basis for MARL settings. However, in this case the agents
must learn about the equilibrium strategies by playing the game, which means they do not have a
priori knowledge of the reward or transition functions. Schwarz considers two properties which
should be used for evaluating MARL algorithm: rationality and convergence. The latter simply states
that the method should converge to some equilibrium whereas the former suggests that the method
should learn the best response to stationary opponents. A similar set of conditions is considered by
Conitzer and Sandholm in \cite{ConitzerAWESOME:}, whose algorithm we will consider shortly. Schwarz
then presents a review of MARL methods (as of September 2014) 

\subsection*{Dec-POMDP and MARL}

As Dec-POMDPs are the theoretical formulation of MARL problems, it stands to reason that other
methods for solving Dec-POMDPs should provide insights into improving MARL. Fortunately Oliehock in
\cite{OliehoekDecentralizedPOMDPs} presents a number of existing methods towards solving Dec-POMDPs.
The methods which I feel may be applicable are  \begin{itemize} \item Alternating Maximisation: This
is effectively coordinate ascent for determining a joint policy. \item Approximation as Bayesian
Games: Perhaps solving the repeated Bayesian Game through MARL would be more efficient. It does,
however, force us to ask how best to sub-divide a Dec-POMDP into a series of Bayesian Games  \item
Selecting sub-tree policies: Could DL be used to determine which sub-tree policies are optimal? In
order to make this work, we would also need to consider how to select the feature space of sub-trees
and how to collect this information. \end{itemize}

\chapter{Multi Agent Control}

\section{Stochastic and Robust Model Predictive Control}

The idea of Stochastic Control is best described in the research statement of the Control and
Analysis of Stochastic Systems (CASS) group at Penn State:

\begin{displayquote}
	"...to model, predict and control uncertain engineering systems where the interplay between
	dynamics and uncertainty (stochasticity) is important."
\end{displayquote}

When operating in the real world, systems are subject to uncertainty and environment perturbations.
Despite this, for their safe operation, they must provide certain guarantees (such as collision
avoidance) and remain stable. It is with this in mind that we must design control mechanisms for the
system which satisfy these requirements, even in the face of uncertainty. Methods in stochastic
control have found their way into controlling power distribution, chemical processes and (of course)
robotics. 

\subsection*{Model Predictive Control}

Model Predictive Control is a long standing paradigm in AI which looks specifically at the problem
of operating real-world agents safely in the face of environmental disturbances. The overarching
idea begins with the assumpion that we have a model of the environment. As an example, in
the case of autonomous vehicles, we have a model of how adjusting the angle of the front wheels will
affect the heading of the car. We then perform a finite horizon look ahead, in which we estimate the
environment state for a few time steps ahead, and generate a policy for this horizon. The agent
performs the immediate action generated by the policy, and then we repeat the process, after taking
measurements of the environment to determine the error in the system. Throughout this process, the
controller (a.k.a. policy) must remain stable, but also satisfy constraints. From our previous
example, a likely constraint would be that the controller will never result in the car going on the
pavement, where it would present a real hazard to pedestrians. It is through these constraints, and
the requirement of stability that the system is required to remain safe throughout operation. To
that end, mathematical proofs of these properties are provided in the literature ensuring that we
can trust in the system's performance.

Deterministic MPC assumes that the environment is completely deterministic and, therefore, assumes
knowledge of the exact nature of the disturbances. This somewhat naive assumption simplifies
computation and so appears often in the literature, as in \cite{Rosolia2018}. However, it does not
provide strong guarantees outside of games and simulations. Robust MPC brings this to a higher level
of abstraction in which system perturbations, though deterministic, lie on a bounded set. Therefore,
we no longer assume the exact nature of the environment and can guard against worst case scenarios.
This is extremely effective in closed environments or for simple tasks, but falters in more complex
environments.

Stochastic Model Predictive Control (SMPC) lifts the assumption made in MPC, namely that
perturbations are deterministic and lie on a bounded set \cite{Mesbah2016}. To accomplish this, we
define chance constraints, for which probabilistic guarantees must be determined. Furthermore,
optimality is defined in terms of the minimisation of the expectation of a probabilistic
cost function. This formulation brings about controllers which are more applicable and robust when
deployed. However, the field is in its infancy and presents a number of theoretical challenges.
These are best described by Mesbah in \cite{Mesbah2016}

\begin{itemize}
	\item The arbitrary form of the feedback control laws
	\item The non-convexity and intractability of the chance constraints
	\item The complexity of the uncertainty propagations
	\item Establishing stability of the control problem
\end{itemize}

Since the review \cite{Mesbah2016} was written in 2016, there have been a number of approaches towards solving
some of
these problems, most notably the intractability of the chance contraints. For instance, in 
\cite{Paulson2019}, Paulson and Mesbah propose the use of joint chance constraints in considering
time varying stochastic disturbances as well as model uncertainty. This is shown to be strongly
suited to non-linear systems, which itself is an open problem in the MPC sphere.


The research in using Distributed Predictive Control is quite sparse as compared with, for example,
single agent predictive control. The added complexity of stochastic environments has meant that this
area of research is still very open for looking into. With this said, the main works that I have
found useful whilst reviewing the literature are \cite{Conte2014}, \cite{Christofides2013} and 
\cite{Giulioni2015}. In almost all of the presented methods, a number of assumptions are made.
Usually these are some combination of the following:

\begin{itemize}
	\item The agents act in a deterministic environment. This suggests that predicted futures will
	be close enough to the ground truth.
	\item The agents are uncoupled from one another. This is a strong assumption to make and
	essentially allows us to consider the agents as completely independent of one another. 
	\item The agents can easily communicate with one another. This allows for longer pieces of
	communication (most often predicted trajectories) to be communicated between agents.
\end{itemize}


\subsection{Directions}

The problem of MPC, both in the stochastic, and the robust sense, are rife for work in theoretical
and applied problems. Control in autonomous systems is also one of the more applicable problems
related to STAI since it requires that we place mathematical guarantees on the performance and
stability of the system as well as guarantees on the satisfaction of constraints. This is, of
course, an ever present problem in autonomous systems which operate in the real world. I begin by
listing the more theoretical areas for exploration in MPC before then considering their application
domains.

\begin{itemize}
	\item Significant exploration is required with regards to the optimisation problem. Especially
	for non-linear (or coupled systems), solving the optimisation problem is a struggle. As such,
	approximate methods (such as sampling) should be explored and their effect on the system
	guarantees presented.
	\item Uncertainty propagation in SMPC is also an important problem since we must be able to
	determine the uncertainty in the environment model in order to generate control laws. We should
	consider a comparison of formal techniques, such as Bayesian inference (e.g. Markov models) or
	particle filters, against modern deep learning methods for propagation.
	\item Adaptive MPC is particularly important when considering failures in a system. In his talk,
	Mark Cannon considers the platooning problem (in which multiple autonomous vehicles move
	together whilst maintaining a specified distance from one another). Consider, then, the case
	where one vehicle undergoes a fault (burst tyre etc). The system model must incorporate this,
	and hence requires adaptation. It would again be interesting to consider the use of neural
	networks in this manner. However, specifically we should consider whether the use of a function
	approximator in the form of an NN might result in a violation of the constraints.
	\item Distributed predictive control. This is what, in my view, we could work towards. This is a
	difficult problem due to the coupled nature of the system dynamics. We will need to consider: the
	effect that each agent has on the other, the effect of the environment on all agents, estimating the
	future behaviour of each agent, and ensuring that all agents do not violate the constraints. It is
	here that I believe game-theoretic notions (especially differential games) will come to our
	aid. Similarly, mean-field games should be considered in the general N-agent problem, as well as
	in the case of multiple disturbances, but I will elaborate on this in the applied problems
	section.
\end{itemize}

the following presents a consideration of the applied end of MPC problems. I have tried to focus on
the domain independent suggestions, since the domain specific ones are virtually limitless.

\begin{itemize}
	\item Mean-field predictive control. Consider the case of a hybrid system, in which many components
	interact with each other. In addition, we may consider that there are multiple known disturbances
	acting on the system. This, at first seems like an impossible problem to solve due to the many
	interactions between all elements of the system. However, we can consider this as a game
	theoretic problem in which the components and the disturbances are players. The relatively new
	class of 'mean-field games' is a method which allows the many body problem to be approximated as
	a series of two-agent problems \cite{Yang2018}. Whilst the concerns in the 'Distributed
	predictive control' remain, this may prove to be a valuable area of applied research.
	\item Swarm control. Swarms often have simpler control rules implemented within each agent,
	which emerge into more complex results. It would be interesting to consider the application of
	robust and stochastic MPC with the aim of improving swarm results and satisfying constraints of
	the emergent phenomenon. This would be of particular interest in the case where the swarm has to
	track multiple objectives. A prediction of the swarm evolution may prove beneficial to ensure
	that all agents comply with the expected behaviour as the swarm switches tasks. Similarly, an
	adaptive stochastic model would be interesting to examine here to remain robust to agent
	failures in the group.
	\item Modular Robotics?
\end{itemize}

\section*{Multi Agent Dynamics}

Multi Agent Dynamics considers the problem of mathematically modelling learning in Multi Agent
Systems (MAS). This model then serves to be able to predict the evolution of a learning system as
well as to understand the trajectory of learning. Typically, this looks at considering whether or
not the method is likely to converge towards a Nash equilibrium. This is generally a difficult
problem to solve \cite{ShohamMultiagentFoundations} for all but toy problems. To extend this
applicability into real world settings requires the study of stable equilbrium points;
\cite{Letcher2019DifferentiableMechanics} shows that the stable equilbrium and Nash equilibrium (NE)
are not necessarily the same and, in fact, argue that stable points are more informative than NEs.
Stability provides some guarantees against the stochastic nature of the environment since a stable
equilibrium will always be returned to even after perturbations. This extremely important in Safe
and Trusted AI as it provides guarantees against undesired behaviour in real world
environments. 

\section{Stability of Learning Agents}

Stability may be looked at from the view point of two perspectives. The first is from an
optimisation point of view. This considers the dynamics of the learning model, allowing us to better
choose our parameters and design our models so that they may converge to a stable result. The second
is from the view point of the state-action space of a learnt model. This allows us to determine,
before the MAS is deployed, which set of state-action pairs will lead to unstable behaviour. This
knowledge allows us to consider which state-action pairs should be avoided. In both cases, stability
analysis allows us to build multi agent systems which will learn and act in the way that we expect
them to.

In \cite{Letcher2019DifferentiableMechanics}, Letcher et al. model gradient descent learning of
generative-adversarial-networks (GANs) as a two-player differentiable game. A differentiable game is
one in which the loss function is twice differentiable. Using this formulation, they are able to
analyse the system from new perspectives by considering the current state-of-the-art understanding
of differentiable game theory. Whilst, at first glance, this may seem like a purely theoretical
exercise, they go on to show that the insights gained allow them to develop a new multi-objective
optimisation technique for GANs which shows stronger convergence properties, most notably of which
is that it guarantees that the method finds a stable equilibrium (and avoids saddles) between the
two players' loss functions.

Jin and Lavaei \cite{Jin2018Stability-certifiedPerspective} consider the policy of a reinforcement
learning agent as a non-linear, time varying feedback controller. Using this notion, they then
consider the bounded-input-bounded-output stability of the system. They do this by analysing the
ratio between the total output and total input energy (called the L2 gain). If the L2 gain remains
finite then the system may be considered to be stable. They then apply these considerations on
real-world applications including multi-agent flight formation and obtain stability certificates
(essentially confirming that the system will remain stable under certain conditions) for the learned
controller.

Berkenkamp et al. \cite{Berkenkamp2017SafeGuarantees} consider a similar problem from a different
definition of stability. Specifically, they look at stability from the point of view of Lyapunov
functions. A system is said to be stable if the applying the policy will result in stricly lower
evaluations of the function. In other words, a system is stable if its corresponding Lyapunov
function is decreasing towards a minimum point. The authors use this idea to define a 'region of
attraction' in which the system is stable in the sense of Lyapunov. The goal of Safe Lyapunov
Learning, a method which they develop from these insights, is to learn a policy without leaving this
region of attraction. They do this by taking measurements within the set and using this to learn
about the system dynamics, thereby increasing the safe set. With this, we now have a guarantee that
policy optimisation will not result in unsafe behaviour, even in the presence of stochasticity and
exploration.

As we can see from the discussion thus far, there are many definitions and perpectives of stability,
each lending to a new understanding of learning systems. To add another to the mix, Milchtaich in
\cite{Milchtaich2007StaticGames}, presents a notion of static stability. This means that it is based
solely on the incentives of the players and does not require a consideration of the dynamics of the
system. Milchtaich's definition of a stable system is one in which, when perturbed, it is more
beneficial for an agent to move back towards the equilibrium than it is for them to move away. This
is perhaps the most fundamental definition of a stable equilibrium and, as such, Milchtaich shows
that it is applicable to all strategic games, within certain assumptions (finite set of player and
continuous strategy space) and considers probabilistic perturbations from the original state. This
is particular applicable to Multi Robot Reinforcement Learning (or any MARL in continuous strategy
spaces) since these systems require random exploration of the strategy space and the dynamics are
not known a priori. However, the lack of dynamics means that we do not consider the evolution of
learning. 

\section{Learning Stable Agents}

The above discussion has considered how notions of stability and control are important to
Multi-Agent Learning. It stands to reason, therefore, that Multi-Agent Learning and, more
specifically, game theory, can be used to develop controllers which drive systems to stable states.
This idea is explored in \cite{Marden2018AnnualControl}. Here, a zero-sum game is considered in
which the players are a controller and an adversarial environment. The design of the controller must
be such that it is able to drive the system to zero error. To illustrate, consider the problem of
designing a controller for a re-entry vehicle, as in \cite{Breitner1994ReentryGame}, in which
vortices seek to destabilise the agent. This will allow us to build stable agents in a much more
efficient manner since we can simulate the adversarial environment and hypothetical scenarios the
agent may encounter without actually encountering them. The same notion is explored by Bardi et al
\cite{Bardi1991DifferentialDisturbances}.

 Mylvaganam et al, in \cite{Mylvaganam2017AutonomousApproach}, consider the N-robot collision
 avoidance problem, similarly from the point of view of differential game theory. They develop a
 robust feedback system for the robots which they show to be able to drive the system towards
 predefined targets whilst providing guarantees of interference from other agents (or lack thereof).
 In \cite{MylvaganamASystems}, Mylvaganam also considers a game theoretic control of multi-agent
 systems in a distributed manner. Here, agents only consider their own payoff structure and have
 limited communication with one another. The author shows that the an approximate equilibrium can be
 found using algebraic methods and illustrate the capabilities of the technique using a collision
 avoidance example. For the sake of brevity, I will not include all of the numerous strides that
 Mylvaganam has introduced to the area. However, I must conclude with those presented in
 \cite{Mylvaganam2014}. Here, the author presents approximate solutions to a number of differential
 games, including linear-quadratic differential games (in which system dynamics are linear functions
 whilst payoff functions are quadratic), Stackelberg differential games, where a hierarchy is
 induced across the players (a notion was suggested in the research proposal) and mean-field games,
 which is discussed in Section \ref{sec::MARL Approaches}. The importance of the linear-quadratic
 differential game is the stability of the solution; solutions for the Nash equilibria are
 admissable iff they are locally exponentially stable (which the author often shows with the aid of
 Lyapunov functions). Approximate solutions to the NE are developed which are more feasible to
 calculate online. The author then shows that this is not simply a theoretical exercise by applying
 the novel methods towards multi-agent collision problems and designs dynamic control laws which
 guarantee that each agent will reach their desired state whilst avoiding collision with the other
 agents. Similarly, the Stackelberg game is applied to the problem of optimal monitoring by a multi
 robot system. 

\subsection{Directions}


I will briefly summarise the research directions which I develop in this section, to save ourselves
some time, and to preserve sanity. \textbf{I'll do this later. Sanity is overrated.}

The above discussion has illustrated a few points. The first, and perhaps most important, is that
stability is extremely applicable to Stochastic Optimal Control. This is because of the fact
that learning agents, especially in real world environments, are subject to uncertainty and
perturbations as well as epsilon-random exploration. As such, it is important that the system is
able to return to a stable state after such perturbations. This leads to the need to develop
learning systems as in \cite{Letcher2019DifferentiableMechanics} which are attracted to stable
equilibria. Similarly, it allows us to consider existing algorithms and whether they lead to stable
equilibria. For instance, Letcher et al show in \cite{LetcherSTABLEGAMES} that the Learning with
Opponent Learning Awareness (LOLA) algorithm \cite{Foerster2018LearningAwareness} does not converge
to stable fixed points.

The second point is that the notion of stability has multiple definitions. Each one is appropriate
in different conditions and, as such, it is important to explore stability from various viewpoints
to consider how to best understand the evolution of a learning system. 

Another factor to consider is suggested in \cite{Milchtaich2007StaticGames}; multiplayer games do
not necessarily contain stable fixed points. The existence of stable equilibria is therefore an open
question and is well worth consideration. This would help guide the choice of loss functions and
strategy spaces. This is likely to be an extremely difficult problem to generalise but would have a
dramatic impact on the safety certification of MARL systems.

The problem of finding stable solutions through multi agent dynamics also opens the door for a large
degree of application domains which are currently quite sparse. I would recommend looking at the
contents page of \cite{Hamalainen1991DifferentialFinland} which illustrates a number of applications
of dynamic and differential game theory in a number of domains including: pursuit-evasion, systems
control and economic modelling. My particular interest, as always, lies within robotic control
problems. This appears to be the most direct and relevant application of stable MAS systems. An
example of this might be the problem of Cooperative Moving Path Following Control considered by
\cite{Reis2019RobustVehicles}. Here, multiple agents must track moving targets according to a
pre-defined path without interfering with one another. It is clear then that each player's control
is dependent on the other and so can be modelled as a game theoretic problem. Our task, then, is to
determine control rules which lead to stable coordination across the team. Similarly, we can model
consider modelling a controller and an adversarial environment as a zero-sum game as examined in
\cite{Marden2018AnnualControl}. 

Another application is proposed by Letcher et al in \cite{Letcher2019DifferentiableMechanics} - that
of Generative adversarial networks (GANs) which can be considered as a two player minimax game with
differentiable loss functions. As pointed out by Mylvaganam, it may also be interesting to consider
the differential game problem from the point of view of different information structures. In
real-world systems, it is unlikely that all agents will have access to the information of other
agent' cost functions and current positions. It could, however, be interesting to explore the use of
localisation methods to determine and track this information through observation (e.g. through
Kalman or Bayesian methods). Finally, most of the relevant work in stochastic optimal control
considers the case where system dynamics are linear. A whole new set of problems arise in the more
realistic case where we must consider the non-linearity of system dynamics. Such a consideration
would also allow us to extend these technique to the more difficult (and important) problems in
control.

The following papers \cite{Bailey2019FiniteDescent-Ascent, Bailey2019Multi-AgentSystem,
Boone2019FromTheory, DickensTheLearning, Berkenkamp2017SafeGuarantees,
Jin2018Stability-certifiedPerspective, Letcher2019DifferentiableMechanics} are the ones that I found
particularly relevant to this study. However, they will require some further analysis before I write
about them here. (Note: I've written about some of these in the paragraphs above, but I'm not
entirely sure which ones yet. I think the job over Christmas will mainly be about sorting this
document out.)

\chapter{Research Directions}

\section{Stochastic Model Predictive Control for Distributed Systems}

The work on distributed stochastic MPC (DSMPC) is still rather sparse but is advancing rapidly. To
that end, I propose the following techniques towards improving DSMPC.

\begin{itemize}
	\item Each agent $i$ takes into consideration their observations of the past N measurements of the
	actions selected by another agent $j$. Using this, a Gaussian Process (GP) regression informs the
	distribution of the action that will likely be selected by $j$, effectively performing a
	one-step lookahead of $j$'s actions. By propagating the effects of this predicted input $i$ can
	then determine a likely trajectory for $j$ and then perform its own update accordingly. Whilst
	this may seem computationally expensive, its advantage lies in the fact that it requires little
	communication across the agents. An immediate disadvantage of this is that, as the number of
	agents increases, $i$ would need to perform all of these calculations for each $j \neq i$. I'm
	not yet confident on the feasibility or the usefulness of this technique, but looking further
	into the literature (which seems to add new things every day) may provide insights which could
	help refine the technique. This method is, in many ways, similar to that presented in 
	\cite{Conte2014} and \cite{Dai2017}. The advantage of this proposal over that in 
	\cite{Conte2014} is that it does not assume that the uncertainties from the predictions of other
	agents are bounded. Likewise, the advantage over \cite{Dai2017} is that our proposal allows for
	all systems to update at each time step.

	\item The second proposal is motivated by the ideas presented in \cite{Foerster} and 
	\cite{Heirung2019}. In this methodology, a centralised controller maintains and updates a joint
	public belief based on the observed actions of all agents. This joint belief will likely provide
	some representation of the trajectories that each agent will carry out or an understanding of
	the other agents' future actions. In a similar manner to \cite{Heirung2019}, this joint belief
	will be chosen from a family of existing models using a recursive Bayesian update. The idea
	would be for this update to be conditioned on the history of observations as is done in 
	\cite{Wingate2012}, though this is more flexible. Agents have access to this joint public belief
	state and use this, in combination with their own system models to determine a feasible
	trajectory. The challenge here would be proving the stability and controllability of the system
	and that it requires communication with a central controller. However, it would appear to be a
	feasible area of exploration.
\end{itemize}

\section{Stochastic Control of Swarms}

Swarm systems and algorithms are a relatively mature area of research. Yet there is work to be done
when considering its application in stochastic environments. In particular, the ideas presented
below concern the transport problem of swarms, i.e. moving a swarm from an initial distribution to a
desired distribution. 

\begin{itemize}
	\item Consideration of state constraints or boundary conditions (terms which I will use
interchangeably) in the diffusion process. In general, stochastic swarm systems are modelled using a
particular stochastic difference equation (the Ito equation). However, this does not account for any
boundary conditions on the diffusion. Indeed most swarms use simple 'stop-and-turn' notions of
obstacle avoidance. However, in environments where there are human operators, or in general areas
where we do not want swarms to enter, it is important to have a guarantee of constraint
satisfaction. To accomplish this, I propose extending the Ito equation to incorporate boundary
conditions. This will, firstly, provide a more suitable model for the evolution of a swarm system in
confined spaces, allowing for a more intuitive understanding of the future behaviour of the swarm.
Furthermore, it may present an opportunity for us to consider the decentralised control of the
agents which can ensure state constraint satisfaction. The resources which are most relevant to this
study (and from which the ideas are derived) are
\cite{Inoue2019,Sartoretti2014,Shahrokhi2019,Li2017}. 

\item Consideration of pairwise interactions between agents to ensure collision avoidance. Most of
the diffusion models, as mentioned above, do not take into consideration the interactions between
agents, since it adds a large degree of complexity to the model. However, it would be important from
a safety perspective to take these into account by looking at the attractive and repulsive
potentials across agents. This would allow us to provide a guarantee of inter-agent collision
avoidance whilst maintaining the diffusive progress of the swarm. In \cite{Inoue2019}, the authors
mention that they are already looking into this problem and have proposed a similar manner of
tackling it as I have. Perhaps we may be able to build upon or improve their results if and when
they publish?

\item Finally, all of the diffusive methods for swarms consider a time invariant network topology.
By this, I mean that it is assumed that each agent can communicate with, or observe, a certain
number of their nearest neighbours. However, this may not always be the case. It will often be that
in different environmental conditions, the observation or communication capabilities of the robots
may increase or decrease. To better accomodate for this fact, I propose the extension of the model
with the addition of Markovian Jump Dynamics (MJD). MJD allows us to take into account discrete
'jumps' in the structure of the system model which occur randomly. It has shown a large degree of
success in stochastic control of multi-agent systems but has, as far as I can see, yet to find its
way into swarm dynamics. It will, however, become particularly important in cases where we can no
longer take the consistency of agent communication for granted. 
\cite{FUHRMANFrancescoRUSSO,Ma2017,Li2017} have begun to make progress in these areas, but it does
not seem like the connection of MJD and swarm dynamics has been made yet.
\end{itemize}

\section{Multi Agent Reinforcement Learning}

The ideas I have down here are a little more sparsely populated, but further reading may provide
some insights into potential directions.

\begin{itemize}
	\item Stable exploration in MARL. This builds on the ideas of \cite{Berkenkamp2017,Jin2018Stability-certifiedPerspective} and
	an important point mentioned in \cite{Marinescu2014} that 'the dynamics implied by multi-agent
	systems lead to stochastic behaviour resulting sometimes in undesired effects'. In particular,
	most RL methods require some trade-off between exploration and exploitation. It is, therefore,
	required that we determine a safe manner in which MARL agents may explore without leading to
	unstable behaviour. This would require extending the work of \cite{Berkenkamp2017} and/or 
	\cite{Jin2018Stability-certifiedPerspective} to the multi-agent case. Note that 
	\cite{Jin2018Stability-certifiedPerspective} does actually consider a multi-agent case, but with
	independent learners who do not consider the effect that they have on the other agent(s). The
	particular extension that I propose is to consider the coupled effect between agents as part of
	the determination of a safe set for exploration.

	\item Improvement of mean-field MARL. This method has shown the most promise for large scale
	multi agent systems since it allows us to reduce a many-body problem to a series of two-body
	problems. Although it makes strong assumptions about the nature of the system, in practice it
	has shown strong results. Particular examples of this are found in 
	\cite{Subramanian2019,Yang2018}. I am not yet sure about how to expand upon these methods but
	perhaps a way to start would be to consider the expansion of mean-field methods to heterogeneous
	systems by considering sub-groups within the population. Each of these may be updated
	individually. It would be interesting to examine how this might increase the complexity, and
	also the generalisability of the mean-field method.
	\item A Dynamical Systems analysis on Games. The ideas here still require further reading on my
	part but the potential directions are largely summarised in Section 2.3.1.
\end{itemize}

\bibliographystyle{IEEEtran} 

\bibliography{IEEEfull,references}

\end{document} 
