\documentclass{article}
\usepackage[margin = 2cm]{geometry}


\title{Characterisation of Dynamics}
\author{Aamal Hussain}
\date{}


\begin{document}
    \maketitle

    The first techniques explored are from the paper: "Characterization of nonlinear dynamic systems
    for engineering purposes – a partial review". 

    \paragraph*{Approach based on 1D time series}

    The fractal dimension tells us how the detail in a pattern changes with the scale at which that
    pattern is measured. It also gives us a reasonable measure for what the 'space-filling' capacity
    of a pattern is. The proposed dimension that this paper offers is 

    \begin{equation}
        D = \frac{log_{10} n}{log_{10} (L/l) + log_{10}n},
    \end{equation}

    where $n = L/u$. Here, L is the total length of the curve and l is the distance between the
    first point of the curve and the furthest from the first, u is the average distance between two
    successive points. This seems like it'll give a reasonable estimate for how crazy the curve is.
    If it goes straight to an equilibrium, it should be close to 1. However, if it is crazy, the
    dimension will be higher than 1.
    
    Other viable options include spectral analysis (which I actually should consider), recurrence
    plots (which the authors don't talk about but is useful for detecting periodicity) and average
    time spent in a small region around the attractor.

    

    I think this actually covers all of the problems that I would need to consider. The main ones
    which I will want to have a look at implementing tomorrow are: the fractal dimension outlined
    above, the recurrence plot and time spent in a region of phase space. In addition, I'd like to
    find something which gives an indication r.e. the spectral analysis problem and then I can put
    everything into code and hopefully get it running with some results.

\end{document}