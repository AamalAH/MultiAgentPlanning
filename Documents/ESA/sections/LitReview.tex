\documentclass[.../main.tex]{subfiles}

\begin{document}

In this chapter, we explore the current state of the art in Multi-Agent Systems (MAS). We will
consider the main techniques that have been proposed towards task allocation and control. 

\section{Multi Agent Systems} \label{sec::Multi_Agent_Systems}

We will keep the problem statement defined in Section \ref{sec::Objectives_and_Scope} in mind as
we define several metrics which a Multi Agent System (MAS) may be assessed against.

\begin{enumerate}
    \item Distributed. It will often be the case that a multi-agent system must spread out across a
    large area or operate in regions where communication is limited. As such it is beneficial that
    such a system does not rely too heavily on communication with a central body. The individual
    agents must, therefore, be able to make their own decisions and act as independent entities.
    Aside from the advantage of communication, this ensures that the system is not tied to a single
    point of failure; in a system governed by a centralised decision maker, any faults in the
    central body propagates throughout the group. Of course, this comes at the price of
    computational load - if the agents are required to act independently, they must possess the
    resources to do so. The term distributed should not be confused with decentralised. The latter
    refers to a regime in which agents all communicate with one another before making a decision.
    Whilst this is preferable to a centralised decision maker in terms of the requirement for one
    entity to communicate with all agents, it still places a high cost on the requirement that
    agents be in constant communication with one another. In fact it may pose the additional risk
    that if any agent in the team fails, the whole pipeline falls apart. Distributivity, on the
    other hand, requires that agents make independent decisions so that, though agents may
    communicate with each other, they are still able to operate when this is not possible.
    \item Generality. MAS come in many different flavours; sometimes a small group of heterogeneous 
    (agents with different capabilities) agents may be required to perform a delicate task whilst
    in others, it may be required that a large population of homogeneous (all of the same type)
    agents perform multiple tasks simultaneously. Indeed the same MAS may have to adapt its
    strategy whilst in the middle of a task. It would be a hindrance if the agents were required to
    be reprogrammed with a completely different control methodology each time the situation changes
    slightly. It would be preferable if the method could apply to all different types of scenarios,
    with homogenous or heteregeneous agents, with single tasks or multiple, in any environment with
    only minor customisation required. Better still if the agents are able to adapt themselves to
    the specifics of the scenario.
    \item Low Computational Load. As mentioned in the previous chapter, a MAS will often be used in
    complex situations, many of which will be too dangerous for human intervention. As such, the
    system will be required to carry out its tasks online. For this reason, it is vital that the
    control method be one which produces low computational load - especially where the time frame
    for adjusting agent behaviour is limited. Monetary concerns also play a part in this metric; it
    would not do to impose that every member of the team be equipped with high performance computing
    power as this would likely fall outside of the budget of any practical usage.
    \item Robustness. Particularly for physical robots, a MAS should be able to operate in
    situations which present sensor noise, communication noise, latency, and environmental
    disturbances. This is rather important in unseen environments where full a priori knowledge of
    the map cannot be assumed.
    \item Scalable. This is mentioned last since the ability for a MAS to scale comes from its
    distributivity  and low computational load. Scalability is the capacity for any method of
    controlling a MAS to successfully operate as the size of the team increases. This, of course, is
    related to the complexity of the algorithms involved. However, it should be noted that it is not
    always required that a MAS be formed of dozens (or even hundreds) of agents. Depending on the
    task at hand (e.g. robotic surgery), it may be beneficial for fewer, more capable agents to
    perform the task, rather than a large population. 
\end{enumerate}

We begin by presenting a subset of the vast array of literature regarding MAS and focus on those
which are relevant to the study at hand. We conclude this chapter by providing some remarks about
how these methods hold against the above metrics.

\section{Game Theory} \label{sec::Game_Theory}

Game Theory has a rich history when considering an understanding of multi-agent systems and it
is impossible to examine MAS without an understanding of Game Theory. These
begin in economics but have found a strong application in computation due to the rising need for
distributed systems. Game Theory, therefore, branches across all of the categories in this
chapter although its synergy with swarms requires development) since Dec-POMDP
and MARL methods have both base their theory upon the foundations of Game Theory.

\subsection{Partially Observed Stochatic Games (POSG)} \label{sec::Stochastic_Games}

POSGs act as a game theoretic corollary to the Dec-POMDP. Here, an optimal solution is found by
fixing the strategy (a manner of selecting an action \cite{Rizk2018}) of all agents
except one. The optimal strategy of the chosen agent is then computed. This is fixed so that the
next agent may determine its optimal strategy. This process is iterated over all agents in a process
known as Alternating Maximisation \cite{Ray2010}. By choosing strategies which maximise
a common payoff, a globally optimal solution can be found.

Similarly Bayesian Games is a game theoretic approach with incomplete information (thus it is
sometimes referred to as an ‘incomplete information game’). This allows for a POSG to be
approximated by a sequence of smaller, more tractable games
\cite{Emery-Montemerlo}. However, iterating through the entire team of agents to
find an optimal solution is a lengthy process and does not allow for immediate actions in dynamic
situations. To address this, \cite{Ray2010} propose to place distinct payoffs on each
agent and to consider team formation based on social hierarchy as well as preferred partners. This
allows the complexity of the problem to be broken down into smaller game units and also provides a
clear order in which the games are to be played. This minimises the interference across robots and
allows for more immediate action to be taken by agents who choose their strategy early on, while
leaving more passive tasks for those later in the sequence. 

The Bayesian Game formulation therefore provides a strong candidate for rapid task re-allocation and
dynamic decision making. However, as found by Dai et. al \cite{Dai2018} this is
conditional on an understanding by each robot of the strategy of others, which in turn requires
adequate communication between robots. Therefore, the game theoretical formulation will fail where
communication is not possible amongst team members. However, the addition of heuristics, such as
deep learning, may be able to advance an agent’s ability to recognise the strategies taken by other
members of its team without the need for significant explicit communication.

% \textbf{Add in any recent literature in these fields}

\subsection{Game Theoretic Control} \label{Game Theoretic Control}

Game theory can often be applied to problems of control theory (particularly where there are
multiple agents) to develop robust controllers which guarantee properties of stability and
constraint satisfaction. 

This idea is explored in \cite{Marden2018AnnualControl}. Here, a zero-sum game is considered in
which the players are a controller and an adversarial environment. The design of the controller must
be such that it is able to drive the system to zero error. To illustrate, consider the problem of
designing a controller for a re-entry vehicle, as in \cite{Breitner1994ReentryGame}, in which
vortices seek to destabilise the agent. This will allow us to build stable agents in a much more
efficient manner since we can simulate the adversarial environment and hypothetical scenarios the
agent may encounter without actually encountering them. The same notion is explored by Bardi et al
\cite{Bardi1991DifferentialDisturbances}.

Mylvaganam et al, in \cite{Mylvaganam2017AutonomousApproach}, consider the N-robot collision
avoidance problem, similarly from the point of view of differential game theory. They develop a
robust feedback system for the robots which they show to be able to drive the system towards
predefined targets whilst providing guarantees of interference from other agents (or lack thereof).
In \cite{MylvaganamASystems}, Mylvaganam also considers a game theoretic control of multi-agent
systems in a distributed manner. Here, agents only consider their own payoff structure and have
limited communication with one another. The author shows that an approximate equilibrium can be
found using algebraic methods and illustrate the capabilities of the technique using a collision
avoidance example. For the sake of brevity the numerous contributions that Mylvaganam has made to
this field is not presented here. However, we will conclude with those presented in
\cite{Mylvaganam2014}. Here, the author presents approximate solutions to a number of differential
games, including linear-quadratic differential games (in which system dynamics are linear functions
whilst payoff functions are quadratic), Stackelberg differential games, where a hierarchy is 
induced across the players (a notion was suggested in the research proposal) and mean-field games,
which is discussed in 'MARL'. The importance of the linear-quadratic
differential game is the stability of the solution; solutions for the Nash equilibria (NE) are
admissable iff they are locally exponentially stable (which the author often shows with the aid of
Lyapunov functions). Approximate solutions to the NE are developed which are more feasible to
calculate online. The author then shows that this is not simply a theoretical exercise by applying
the novel methods towards multi-agent collision problems and designs dynamic control laws which
guarantee that each agent will reach their desired state whilst avoiding collision with the other
agents. Similarly, the Stackelberg game is applied to the problem of optimal monitoring by a multi
robot system.   

\section{Multi Agent Reinforcement Learning} \label{sec::MARL}

Reinforcement learning extends the Markov Decision Process problem by considering the case where the
reward model is not initially known to the agent. In a similar manner, Multi Agent Reinforcement
Learning (MARL) extends the Markov Game setting to one where the payoff structure is not a priori
knowledge. 

The task of MARL is to determine an optimal joint policy for all agents across the game. This joint
policy may be the concatenation of all the individual policy or it may just be options for each
agent to take. In either case, optimality is defined through the standard notions of Nash equilibria
and so, in this section, I will try to consider the broad spectrum of methods which attempt to
achieve this Nash equilibria. The largest problem in MARL is the non-stationarity of the environment
\cite{Hernandez-LealA}. In single-agent settings, it is assumed that the environment is Markovian.
However, this must be lifted in the Multi Agent setting since other agents in the environment will
be learning concurrently. As such, we must now consider that the policy for any one agent will
depend on the policy of all other agents. 

This chapter begins with a selection of the foundational methods which were developed towards
solving the MARL problem. The interested reader may find additional methods and implementations of
these techniques in \cite{SchwartzMulti-agentApproach}.

\subsection{Learning in Two Player Games} \label{sec::Two_Player_Games}

The most fundamental method to learning in Matrix games is the simplex algorithm. This is a popular
method of linear programming (in which constraints are linear). This will be important in
considering more current methods. A similar consideration is given to the infinitesimal gradient
ascent algorithm, in which the step size converges to zero. This method guarantees that, in the
infite horizon limit, the payoffs will converge to the Nash equilibrium payoff. Note that this does
not necessarily mean that both agents will converge to a single Nash equilibrium. This is a
particular problem in games where there are multiple Nash equilibria. However, in practice it is
difficult to choose a convergence rate of the step size and, without an appropriate choice the
strategy may oscillate as shown in the book. To address this, a modified approach is presented by
Bowling and Veloso which incorporates the notion of Win or Learn Fast (WoLF) to produce WoLF-IGA
\cite{Bowling2002}.
WoLF is a notion we will come across often in MARL and is shown by the authors to converge to always
to a NE. The concern with WoLF methods, however, is that it requires explicit knowledge of the
payoff matrix (which is not so much of a problem for model based methods) and the opponent's
strategy (which is more of a problem in real-world methods). Finally, the Policy Hill Climbing
method (PHC) is shown to converge to an optimal mixed strategy if the other agents are stationary
(i.e. are not learning). However, it is shown that, when this is not the case, the algorithm again
oscillates. The WoLF-PHC adaptation of this method is shown to converge to a NE strategy for both
players with minimal oscillation. 

\subsection{Learning in Stochastic Games} \label{sec::Learning_Stochastic_Games}

Stochastic Games (or Markov Games) form a basis for MARL settings. However, in this case the agents
must learn about the equilibrium strategies by playing the game, which means they do not have a
priori knowledge of the reward or transition functions. Schwarz considers two properties which
should be used for evaluating MARL algorithm: rationality and convergence. The latter simply states
that the method should converge to some equilibrium whereas the former suggests that the method
should learn the best response to stationary opponents. A similar set of conditions is considered by
Conitzer and Sandholm in \cite{ConitzerAWESOME:}, whose algorithm we will consider shortly. Schwarz
then presents a review of MARL methods (as of September 2014)

% \textbf{Add in more recent literature r.e. learning in stochastic games and the robustness of
% MARL}

\subsection{Multi Agent Learning Dynamics} \label{sec::MARL_Dynamics}

Multi Agent Learning Dynamics (often referred to as Game Dynamics or Learning Dynamics)
considers the
problem of mathematically modelling Multi Agent Systems who adapt through repeated interact with one
another. This model then serves to be able to predict the evolution of the system as well as to
understand the trajectory of learning. Typically, this looks at considering whether or not the
method is likely to converge towards a Nash equilibrium. This is generally a difficult problem to
solve \cite{ShohamMultiagentFoundations} for all but toy problems.
\cite{Letcher2019DifferentiableMechanics} shows that the stable equilbrium and Nash equilibrium (NE)
are not necessarily the same and, in fact, argue that stable points are more informative than NEs.
Stability provides some guarantees against the stochastic nature of the environment since a stable
equilibrium will always be returned to even after perturbations. This extremely important in Safe
and Trusted AI as it provides guarantees against undesired behaviour in real world environments.

The area of dynamics which has shown most promise in Multi-Agent Reinforcement Learning is that of
evolutionary dynamics. This draws from the principles of Evolutionary Game Theory (EGT) which
considers similar assumptions to that of MARL: agents are no longer required to be rational and
play the game optimise their expected return through repeated play. Importantly, players have no
knowledge of the others' payoffs \cite{Tuyls2006AnGames}. In \cite{Tuyls2006AnGames}, Tuyls et al.
determine the relation between the replicator dynamics concept of EGT (a differential equation
defining the evolution of the proportion of a subgroup in an evolving population) and Q-Learning
using Boltzmann probabilities as Q-values. The result was a dynamics equation which describes, for
each action, the evolution of its selection probability which could even account for random
exploration. There have since been a number of works which apply the same insight into different
game types and MARL algorithms. In \cite{Bloembergen2015}, these are broken into the following
categories

\begin{itemize}
    \item Stateless games with discrete actions. Here, stateless refers to the idea that the game is
    static and so the environment has no impact on the result. The aforementioned result 
    \cite{Tuyls2006AnGames} fits into this category.
    \item Stateless games with continuous actions. These consider more realistic MARL than the
    previous category by replacing each agent's strategy vector with a probability density function 
    (pdf) over a continuous action space.
    \item Stateful games with discrete actions. This mostly considers stochastic games, where there
    are multiple states with probabilistic (usually Markovian) transitions between them. However,
    extensive form games, which considers more complex phenomena such as sequential moves and
    imperfect information, are also briefly mentioned.
    \item Stateful games with continuous actions. This is one of the more realistic assumptions
    considered. However, the authors point out that this area is yet to see results, leaving it open
    for possible research.
\end{itemize}

Though the above have seen success through experimental validation in games with 2 players and (in
general) 2 to 3 discrete actions, recent work has begun the consideration of improving the models
towards more complex scenarios with larger agent populations. A recent example of this is 
\cite{Hu2019}, in which Hu uses a mean field approximation to model the dynamics of a population of
Q-Learning agents. As a reminder, mean field (MF) approaches in MARL consider that an agent updates
its strategy based on the mean effect of the population. The result is a system of three equations
which describes the evolution of Q-values over a large population in a symmetric bi-matrix game.This
presents an important first step in modelling the learning dynamics of large agent populations and
has the scope to be expanded to systems of asymmetric games, heterogeneous populations and stateful
games.

An advantage of determining the evolutionary dynamics of learning is that it can describe the
expected behaviour in different game settings. This is particularly important to understand the
convergence of the methods; certain games will often show cyclic behaviour even with the existence
of a strict NE. For instance, Imhof et al. show in \cite{Imhof2005} that a repeated prisoner's
dilemma game results in cyclic behaviour when considering the options of cooperation or defection.

\section{Control Theory} \label{sec::Control_Theory}

The control theoretic perspective considers generating a set of control laws for the system. These
are chosen with the aim to satisfy certain properties. The main properties are

\begin{itemize}
    \item Stability, that a system will return to the desired setpoint (or within a neighbourhood) if
    perturbed.

    \item Robustness, that a system will perform its function in the presence of uncertainty and
    noise

    \item Optimality, that the system will achieve some defined function 

    \item Feasibility, that the controller will always be able to generate a control law which
    satisfies the desired properties.
\end{itemize}

There are a number of approaches towards control systems. However, the interest of this review lies
in multi-agent systems which must operate in the face of uncertainty. As such, we focus on
stochastic and distributed control. The literature of the control communnity is vast and can be
divided into a number of sub-fields. We consider Distributed MPC (DMPC) and Swarm Optimal Control as
those which are particularly relevant to this study and review them in this section.

\subsection{Distributed Model Preodictive Control} \label{sec::Distributed_MPC}

Rawlings et al \cite{rawlings2017model} divide the problem of distributed control into four distinct
categories: decentralised control, non-cooperative control, cooperative control, and centralised
control. The last of these is not considered in the text since it only considers the case in which a
centralised controller has access and can manipulate multiple agents at once. 

Decentralised control is the scheme in which agents do not have information about the actions of
other agents and can only optimise for their own objective. This has the advantage of requiring no
communication, but can often lead to poor performance when the agents are strongly coupled as each
agent’s model is incomplete. In the non-cooperative setting, each agent optimises their own loss
function whilst treating the others’ actions as a known disturbance. In this setting, each agent has
knowledge of the others’ control laws and their effect. In this case, the agents must communicate
their intended actions to one another and iterate to achieve a consensus (or Nash equilibrium). This
is the same for cooperative control, except that the loss function is now shared across the team. 

Of all of these systems, cooperative control has found greatest applicability in autonomous systems
\cite{Negenborn2014}, perhaps due to the favourable stability properties \cite{Venkat2006}. One
particularly strong application of this system is in vehicle platooning. Here, self-driving cars or
unmanned aerial vehicles (UAVs) must move in a certain formation without colliding into one another.
A number of examples can be found such as in \cite{Liu2019, VanParys2017, Zheng2017}. Distributed
MPC provides the advantage that guarantees can be placed regarding coupled constraint satisfaction
and feasibility.

The advances in stochastic and robust MPC, however, do lend themselves towards revisiting the
capabilities of decentralised control. Recall that, in this scheme, the agents cannot communicate
with one another. However, advances in MARL show us that this disadvantage can be made less
prevalent if each agent has a model of the other \cite{Foerster}. To this end, a methodology such as
presented in \cite{Heirung2019} may prove beneficial in the decentralised scheme. Here, the system
chooses amongst a family of system models when choosing its control laws. Similarly, agents who
determine (or perhaps learn) models of the other agents may be able to leverage this information,
minimising the model error when optimising.

% \textbf{Include recent updates in here}

\section{Swarms} \label{sec::Swarms}

Swarm systems comprise of a population of (typically) homogeneous agents who are able to organise
themselves
into formations using a series of simple local interactions with their neighbours
\cite{Couceiro2015}. Whilst the individual agents are generally simplistic, the collective behaviour
may exhibit complex phenomena emulating systems observed in biological organisations such as bee or
ant colonies \cite{Sethi2017}. Hybrid algorithms such as in \cite{Gao2018} show an accelerated
performance in reaching globally optimal solutions in search-based tasks. The advantage of many
swarm algorithms is that they are based on local interactions and so are incredibly scalable
\cite{Rizk2018}.

\subsection{Approaches to Swarm Control} \label{sec::Swarm Control}

Recent work has seen the advancement of swarms controlled in stochastic environments. This is
particularly important for swarm intelligence in robots; many systems are developed with the
motivation of search-and-rescue, in which the robot swarms will have to operate in environments
where accounting for uncertainty is critical. To this end, swarm systems have seen the advent of
stochastic control. In this, the swarm is modelled as a diffusive system using a stochastic
equation, most often the Kolmogorov Forward Equation \cite{Hamann2008}. In \cite{Hamann2008}, the
author shows that this equation can be derived by considering local trajectories of smaller subsets
of the swarm. Elamvazhuthi and Berman \cite{Elamvazhuthi2019} extend this idea to other stochastic
models for swarms, importantly considering an advection-diffusion-reaction model which allows for
hybrid agents to switch between different modes of operation. 

The above models come under the term ‘mean-field models’ \cite{Elamvazhuthi2019b}, a series of
equations for stochastic forward processes (such as swarm foraging) in which, as the number of
agents tends to infinity, the true macroscopic motion of the system tends to these equations.
Importantly, however, these stochastic equations allow for an analysis of the swarm, as well as the
ability to develop control laws. In \cite{Li2017}, Li et al show that these models can be used to
develop control laws for robots in a swarm and drive them towards a target distribution. The method
is shown to perform accurately both in simulation and on real robots. The advantage is that
guarantees can be placed on the convergence and stabilisabiilty of the swarm towards the desired
distribution. However, there appears to be significant scope to expand upon this methodology from a
safety perspective. To begin with, the method does not consider inter-agent interactions and
therefore does not formally guard against collisions. In \cite{Inoue2019}, a similar problem is
considered, although collisions are avoided by having the robot simply move in the opposite
direction when encountering an obstacle. It is here that the mean-field models are not as strong
since they do not take these local interactions into account. It must be noted, however, that Inoue
et al have identified this problem and are currently working towards its incorporation into the
model.

It would, therefore, be of particular interest from a safety perspective to consider agent
interactions. A starting point may be \cite{Bellomo2017} which extends the dynamical model of a
swarm system to include agent interaction. This presents a first step towards considering the swarm
as composed of intelligent agents rather than mindless particles and, as the authors suggest,
presents the possibility of applying game theoretic approaches towards swarms, which also gives the
ability to consider heterogeneous swarm systems who interact with one another through repeated play.
Similarly, it would allow for a stronger control perspective on swarm systems such as presented in
\cite{Borzi2015}, in which a model predictive control (MPC) scheme is presented for a
leader-follower swarm system to achieve given tasks. 


\subsection{Decision Making in Swarms} \label{sec::Decisions_in_Swarms}

% \textbf{Include Marco Dorigo and Recent AAMAS literature in here}

The reduction of the complexity in interactions between agents also allows for the robots to perform
other calculations on board. In \cite{Pini2011TaskSelection}, Pini et al. leverage this by
considering adaptive task partitioning across swarms. This allows a swarm, in a decentralised
manner, to deliberate whether to partition a task into its sub-tasks or to perform the task in its
whole. As of now (to the best of my knowledge) the problem of partitioning general tasks into its N
sub-tasks is unexplored. This, however, highlights another advantage of swarm systems; they are
readily divided into sub-groups (as in \cite{Zahadat2016DivisionInhibition}) to perform a
divide-et-impera (divide and conquer) approach to solving problems \cite{Pini2011TaskSelection}). 

Furthermore, swarm systems may be designed in a leaderless manner and so do not require the use of a
central controller \cite{Couceiro2015}. This presents the advantage that the system can rapidly
adapt to the loss of agents or separation of groups throughout the task. However, the assumptions
made regarding the homogeneity of individual agents and the simplicity of their local interactions
result in significant limitations placed on the complexity of the tasks that swarm systems can
accomplish.


\section{Remarks} \label{sec:remarks}

Each of the above methods present different advantages in their approach towards MAS, as well as a
particular set of disadvantages. 

Game Theoretic methods have long provided a rigorous approach for
modelling the behaviour of any MAS. It places no limitation on the heterogeneity of the agents or
the task demanded. An appropriate choice of payoff matrices will account for both of these
specifications. Recent advances in POSGs have lifted the strong assumptions on prior knowledge that
was required to achieve a Nash Equilibrium and Game Theoretic Control allows for such methods to be
applied in continuous state spaces. Whilst the latter shows particular promise, game theoretic
methods still fall short in terms of distributivity as they are heavily reliant on communication
with a centralised decision maker to supply rewards, or within the team to determine each others'
actions. They also require a priori knowledge of the situation in order to set up payoff matrices
and so cannot be regarded as robust to environmental uncertainty.

Multi-Agent Reinforcement Learning builds on the latter two shortcomings of game theory by allowing
the environment to supply rewards to agents who adapt their behaviour accordingly. MARL is proving
to be a powerful method to control MAS, and is rapidly improving, but cannot yet be regarded as
robust. Studies repeatedly show that small alterations or noise in the environment can result in
large deviations in behaviour. Furthermore, few guarantees can be placed on the resulting behaviour
of a MARL system, though progress is being made in closing this gap.

Control Theory, on the other hand, takes pride in the guarantees placed on the MAS such as:
controllability, stability, and robustness. However, control techniques are often centralised or
decentralised. Distributed methods of control is still an evolving field, particularly in terms of
robustness to uncertainty. In order to establish guarantees, assumptions are also made on the
specifics of the MAS, and so control theoretic methods have yet to evolve into general,
heterogeneous systems with any given task.

Swarm systems have their strongest advantage in scalability and, in fact, swarm techniques are built
to scale due to the distributivity and low computational load placed on each agent. This has often
come with the assumption that agents in the swarm act as simple, homogeneous, particles who follow
very simple rules of interaction. Recent work is showing that this is not a necessary assumption
and, in fact, swarms can leverage heterogeneity and agent capabilities to ensure robustness and
generalisation.

It is clear that each method presents its own set of advantages and disadvantages. In fact, the
greater contributions are made from a unification of techniques (e.g. game theoretic control) which
leverage the strengths of one technique to mitigate the drawbacks of another. To this end, we
consider the unification of swarm control and multi-agent learning. The aim is to unify the
strengths of swarm theory, MPC and game theory to study populations of agents who can make
decisions on an individual basis but perform tasks through large scale collaboration. In the
subsequent chapters, we propose a line of research which explores the properties of a system as well
as establishing practical control methodologies along the way.

\end{document}