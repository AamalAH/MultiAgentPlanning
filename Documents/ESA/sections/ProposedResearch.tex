\documentclass[.../main.tex]{subfiles}

\begin{document}

	This chapter describes a proposed line of research which aims to study the evolution of swarm
	systems with independent, learning agents under the influence of predictive controls. We detail
	the novel lines of research which this work aims to undertake, providing suggestions of the
	hypotheses posed by these lines. With this in mind the following studies are proposed and
	developed subsequently:

	\begin{itemize}
		\item \textbf{Stability and Chaos in MARL:} in which we seek to understand the dynamics of
		agent strategies when using Q-Learning to learn a game through iteration. We will establish
		the stability, as a function of parameters, when learning general p-player N-action games.
		This study enables the appropriate selection of parameters and payoff matrices to ensure the
		stability of the strategies of a finite set of agents, such as leaders in a swarm.
		\item \textbf{Dynamics of Mean-Field Q-Learning Games:} in which we examine the
		strategy dynamics for large populations of agents learning through iterated games and
		mean-field Q-Learning. We seek to understand the long term behaviour of such mean-field
		systems in terms of its strategy selection. This study extends the previous and allows for
		the stability of the strategies of a population of agents to be established.
		\item \textbf{Model Predictive Control of Active Particles through Fields:} in which we
		investigate the interaction of a swarm of active particles with potential fields. We
		establish the stabilisability of an MPC scheme with defined stage costs, as well as an
		analysis of the suboptimality of the method.
		\item \textbf{Incorporation of Intelligence in Control:} in which we adapt the dynamical
		system from the above point to include an interaction term, accounting for strategy
		selection by agents who learn through iterated games against one another. Interactions with
		an MPC scheme is then to be examined in a similar capacity.

	\end{itemize}


	\textbf{TODO: Ensure all references here are in the Lit Review}

    \section{Stability and Chaos in MARL} \label{sec::Chaos_in_MARL}

    This segment of research is aimed towards a deeper understanding of the strategy evolution
    of agents following a Q-Learning approach. This allows for guarantees to be placed on the
    behaviour of such agents, in particular the conditions under which the game will converge to a
    stable equilibrium.

    It has long been established that, upon lifting the strong assumptions made by traditional
    game theory (such as the rationality of agents and complete information), that player
    strategies can result in much more complex behaviour than convergence to a Nash Equilibrium 
    (NE). In fact, this is even true on what would commonly be regarded as 'simple' games such
    as tic-tac-toe and prisoner's dilemma \cite{Galla's work and I think Sato's work}. These
    behaviours include: convergence to a unique equilibrium (though not always to an NE),
    convergence to one of multiple equilibria, limit cycles and chaos. These behaviours are shown in
    Figure ... Of these, the most preferable is, of course, convergence to a unique equilibrium,
    although it is still possible to study systems with multiple equilibria or limit cycles 
    \cite{One of the textbooks}. However, it would be difficult to control systems whose dynamics
    are governed by chaos (though research into controlling chaos is ongoing and rife with
    opportunity \cite{CyberneticalPhysics}) and so MARL techniques should avoid this. It would,
    therefore, be a useful endeavour to determine the conditions under which these sorts of
    behaviours arise.

    The behaviour of a system may be studied given a model of its dynamics. It is through this
    process that a wide array of physical systems, from harmonic pendulums to geophysical fluids,
    can be understood. A growing body of research aims to understand multi-agent reinforcement
    learning through the lens of its dynamics. In this light, Tuyls et al. \cite{Tuyls2006AnGames}
    were able to derive a model of the strategy evolution of agents learning through iterated games.
    Through this, they were able to arrive at the following model of Multi-Agent Q-Learning

	\begin{subequations}
	\label{eqn::EOM}
		\begin{equation}
			\frac{\dot{x}(t)}{x(t)} = \alpha \tau (\sum_{j} a_{ij} y_j - \sum_{i j} x_i a_{ij} y_j)
			+ \alpha \sum_j x_j ln(\frac{x_j}{x_i}) 
		\end{equation}
		\begin{equation}
			\frac{\dot{y}(t)}{y(t)} = \alpha \tau (\sum_{j} b_{ij} x_j - \sum_{i j} y_i b_{ij} x_j)
			+ \alpha \sum_j y_j ln(\frac{y_j}{y_i}).
		\end{equation}
	\end{subequations}

	Here, $\alpha$ and $\tau$ are the parameters of the agent; Sanders et al. refer to these as the
	memory and intensity of choice parameters respectively. Agent 1 takes action i with probability
	$x_i$ while Agent 2 takes action j with probability $y_j$. If these actions are taken, the agents
	receive payoff $a_{ij}$ and $b_{ji}$ respectively. With these equations, it is possible to
	predict the expected behaviour of Q-Learning agents, as shown in Figure ...

	It is clear, both from (\ref{eqn::EOM}) and Figure ... that the long-term strategy selection of
	these agents is determined by the parameters $\alpha, \tau$ and the payoffs $a_{ij}, b_{ij}$. We
	then pose the question: how do these elements influence the types of behaviours seen during
	learning on an iterated game? In other words, under what parameter selections are we likely to
	see convergence to unique equilibria, multiple equilibria, limit cycles or chaos?

	With this in mind, the intention of this area of study is to consider the analysis presented in
	works such as \cite{Sanders2018} and \cite{Galla2011}. Here, the authors examine the
	Experience Weighted Attraction (EWA) algorithm, which is regarded as a strong model for the
	learning behaviour of human players in a game \cite{whatevertheycite}. The authors are able to
	determine the regions of parameter space in which complex behaviour,
	including cycles and chaos, predominantly occur and those in which a given game converges to a
	stable equilibrium. Figure ... reproduces the graphs shown in \cite{Sanders2018} which
	illustrates the successful derivation of a 'phase line', across which learning shifts from
	convergent to chaotic.

    We propose to bridge the analysis presented by Sanders
    et al. and Galla et al. towards games learnt using MARL, particularly Q-Learning with
    Boltzmann exploration as considered by Tuyls et al. This would allow for a characterisation of
    the expected resultant behaviour under given parameters for a particular game. The vision for
    this is to provide guarantees on the stability of a finite set of agents within a swarm, such as
    leaders in a flock. Without this, it would be impossible to ensure the stability of a swarm in
    which intelligent, social interactions must be accounted for. However, this analysis
    would also allow for a characterisation of the conditions under which MARL algorithms may
    feasibly be applied, thereby supporting the ability of researchers and engineers to choose their
    payoff matrices and parameters accordingly.

    This segment of research is currently underway and approaching completion. The analysis can be
    seen in Section \ref{sec::Chaos_in_Q-Learning}.

    \section{Large Agent Dynamics} \label{sec::Large_Agent_Dynamics}

    This segment of research aims to model the game dynamics of large populations of agents
    learning through iterated games and mean-field Q-Learning. The aim is to provide similar
    guarantees of stability as in the previous sections with the caveat that all agents in the
    population can learn, rather than a finite subset.

    One of the main results shown by Sanders et al. \cite{Sanders2018} is that as the number of
    players in a game increases, the learning behaviour is more likely to be chaotic, regardless of
    the choice of parameters. This is intuitive since a higher number of players would result in a
    greater strategy space and more agents for any particular player to learn against and is
    verified by their presented results as in Figure ...

    Yet it can be argued that, for large populations of agents (e.g. a crowd), the aggregate
    behaviour may be predictable. This intuition is the foundation upon which crowd dynamics and
    flocking systems are based. In fact, the work presented by Leung et al. \cite{Hu2019} provides a
    cursory verification of this intuition. Here, the authors present an anaysis of the learning
    dynamics for a large agent population (which they approximate as containing infinite agents)
    where each agent is an independent Q-Learning using Boltzmann action selection. The
    result is a system of equations, governed by a Fokker-Planck model which is numerical shown to be a 
    strong predictor of the overall strategy selection of the population. Indeed, rather than
    examining the strategies of every agent in the population (which the authors approximate as
    infinitely large), they model the evolution of a probability density function which, for each
    action, defines the density of agents who retain a given Q-value for that action. This is
    illustrated in Figure ...

   	As the authors point out, this is the first attempt at considering such a problem, and relies on
	heavy assumptions. The strongest of these is placed on the game itself, which is always
	assumed to be cooperative (in Sanders' terms, $\Gamma = 1$). We, therefore, propose extending
	this model to accomodate an arbitrary choice of games. 

	Two proposals for how this might work

	\begin{itemize}
		\item Estimate the population state using a Random Finite Set (RFS) and apply the models
		from the previous study. This is a new hypothesis so more research is required.
		\item Decompose a generic game into a weighted sum of a competitive and a cooperative game.
		Treat these as two separate agents and invoke the mean field approximation so that any given
		agent is effectively engaged in a three-player game where the strategy of the opponents is
		the average strategy of the population.
	\end{itemize}

	
	\textbf{TODO: Elaborate on the technical details of the RFS and three body hypotheses}


    \section{Swarm Control through Fields} \label{sec::Swarm_Field_Control}

    Here we aim to study the interaction of a swarm of 'active particles' with a potential field. In
    particular this field is to be generated by 'field particles'. A model predictive control 
    (MPC) scheme
    is to be devised to drive this system to desired configurations, with guarantees placed on
    stability and satisfaction of input constraints. The sub-optimality of the control
    scheme is to be studied.

    As described in the previous chapter, the control of swarms has been examined through the use of
    mean field models. These models resolve the fact that it is impossible to view the entire system
    as simply the sum of all of the agents and rather model the overall state of the system through
    density functions. Controls are then applied to this function whose evolution is described
    through, typically, one of two partial differential equations: the Fokker-Planck Equation (also
    referred to as the Kolmogorov Forward Equation) and the Vlasov Equation. The former has seen
    some success in recent literature \cite{Zhang2018, Elamvazhuthi2019, Li2017,
    BorziCollectiveMotion}, though it makes the assumption that the system evolves through 'drifted
    brownian motion'. This means that the agents are considered to move about independently and
    randomly, though under the influence of a field which affects their velocity. This technique
	has been shown, both theoretically and experimentally, to drive swarm systems in a stable
	manner \cite{Any paper of Fokker Planck control}. The latter is beginning to show promise 
	though control is typically through leadership rather than fields \cite{VlasovMPC}. 

	The hypothesis of this section is that the distribution of active particles, whose evolution is
	described through a Vlasov equation, may be controlled through the influence of a scalar field
	generated by 'field paricles'. This is drawn from the dynamics posed by Bellomo et al. 
	\cite{Bellomo2017}

	\begin{equation}
	\label{eqn::Vlasov}
    \begin{split}    
        \partial_t f + \Vec{v} \cdot \nabla_{\Vec{x}} f + \kappa \nabla_{\Vec{v}} \cdot (F_a (f) f)
        = \epsilon Q(f, f) \quad  (\Vec{x}, \Vec{v}) \in \Omega[f] \times D_{\Vec{v}}, \quad t>0, \\
        F_a[f](t, \Vec{x}, \Vec{v}) = - \int_{\Omega [f] \times D_{\Vec{v}}} \psi (|\Vec{x} - \Vec
        {x^*}|)(\Vec{v} - \Vec{x^*}) f(t, \Vec{x}^*, \Vec{v}^*) d\Vec{x}^* d\Vec{v}^*, 
    \end{split}
    \end{equation}


    Here, $f = f(t, \Vec{x}, \Vec{v})$ is the one-particle probability density function at
    phase-space position ($\Vec{x}, \Vec{v}$), $\Vec{v}$ and time $t$ which represents the state of
    the system. $\kappa \geq 0$ is a scalar
    coefficient, $\psi$ denotes the communication strength between particles, and ($\Vec{x}^*, \Vec
    {v}^*$) gives the position in phase space of a 'field particle'. It can be seen that the these
    field particles influence the velocity of the swarm agents.

	The contribution that this study presents over those using brownian motion is the inclusion of an
	interaction operator $Q(f, f)$. This governs how agents interact with one another. With this
	term included, the dynamics account for the tendency of agents to avoid collisions. However,
	the term is intended to account also for social interactions between agents, which we aim to
	leverage in the subsequent section.

	The complexity of this interaction term will need to be gradually increased over time. To begin
	with, we may entirely neglect this term. It is here that we will compare the model proposed by Zhang
	with that proposed by Bellomo et al. We may then consider only local interactions, which is typical
	for most swarming systems in the current literature, before then considering non-local interactions.
	This provides the scope for swarms to interact with a greater number of agents. At this stage, we
	may expand our study to consider the presence of interaction domains. This places constraints on how
	agents may interact with one another, allowing for a greater generalisation of inter-agent
	interactions. Ultimately, we would like to influence the interaction term, though this is discussed
	in Section \ref{sec::Intelligence_in_control}. The questions we will
	consider here are:

	\begin{itemize}
		\item The stablity of the system: Under what conditions is it possible to drive the
		swarm from one configuration to another in a stable manner?
		\item The guarantees of the system: Is it possible to ensure that controls remain within an
		admissible set (typically a closed, bounded and convex set \cite{PDEControl}).
	\end{itemize}

	These results will be established theoretically alongside the sub-optimality of the MPC scheme
	and verified through numerical simulations in a 2D environment. It
    will be of interest to perform a similar analysis to Ko and Zuazua \cite{Ko2019} in which the
    cost functional is altered to favour particular metrics  (e.g. running cost, control time etc.)
    and also analyse the effect of varying the time horizon.

    \section{Incorporation of Intelligence in Control} \label{sec::Intelligence_in_control}

    This section of the study is perhaps the strongest extension proposed in this
	chapter, and will likely be the most challenging. Here, we examine the dynamics (
	\ref{eqn::Vlasov}) in which the interaction term accounts for the social dynamics of the
	decision making of individual agents who have learnt through an iterated game. We leverage the
	strategy evolution dynamics and stability analysis established in 
	\ref{sec::Chaos_in_MARL} and \ref{sec::Large_Agent_Dynamics}. 

	Two proposals for incorporation of social interaction:
	\begin{itemize}
		\item Addition of the agent decisions as a constant in the state $f$. This should be fine if
		it's fixed but note that the controls have no influence over this.
		\item Multiplicative term in the turning probability which acts as a weighting term where
		weights are equal to the agent strategy. This extends to both the swarms in
		which the leaders are making decisions (only leaders have this term) as well as in
		swarms where the population makes decisions (everyone has this term). This will need to be
		done with care though to ensure that normalisation holds.
	\end{itemize}

	The MPC scheme from the previous section will then be adapted to this dynamical system with the
	same questions of well-posedness and stability considered. 

	A possible extension of this would be to consider online learning by incorporating the time
	evolution of the strategy space within the social interaction term, which would now be
	time-varying. The proper integration of these dynamical systems would depend on the form of the
	systems derived in Section \ref{sec::Large_Agent_Dynamics} and it is likely that relaxations would
	be required in terms of controllability results. A likely relaxation would be to examine only the
	behaviour in the long-enough time frame, which means that we could not impose a final time. Of
	course, this means that we could only consider optimal control rather than model predictive.
	However, without establishing the models from Section \ref{sec::Large_Agent_Dynamics}, this
	extension is speculative and it is likely that such an extension would fall outside the scope of
	the PhD.

	\textbf{TODO: elaborate on the technical details for proposals of social interaction, mention
	the conservation laws}

\end{document}