\documentclass[.../main.tex]{subfiles}

\begin{document}

    Swarm robotics bases its success upon the collaboration of many agents who contribute towards
    the overall success of the system \cite{Hamann2018}. The collaboration of many agents who share
    the same goal can give rise to behaviours that are far more complex than individual robots can
    achieve. The capacity for complexity is becoming increasingly important as robots are we are
    presented with ever more challenging tasks including: search and rescue, cleaning up space
    debris, and robotic assembly. Such tasks are beyond the capabilities of individual robots and so
    we must leverage the capacity for agents in a swarm to collectively achieve our goals.
    Fortunately, a growing subset of the Control Theory community have turned their attention
    towards this very problem and address the question: how does the dynamical behaviour of a
    population of agents change under the influence of control? Note that we use the term `agent'
    loosely here, an agent can refer to any entity to which controls may be applied, from a simple
    particle \cite{Roy2017} to a complex robot \cite{Elamvazhuthi2019}. In answering this question,
    researchers have gained a stronger understanding of how a swarm may be controlled to achieve
    desired results.

    Another technique which has made incredible strides towards improving the capability of
    autonomous systems to communally achieve results is Multi-Agent Reinforcement Learning (MARL)
    \cite{SchwartzMulti-agentApproach}. This builds on the ideas of classical Reinforcement Learning
    which allows an agent to optimise its strategies towards tackling a task through repeated
    exposure to that task. MARL utilises this same principle with the caveat that agents must adapt
    their strategies based not only on the outcome of the action, but also on the actions of the
    other agents. This has shown strong results in practical applications \cite{Woolridge2009,
    SchwartzMulti-agentApproach, Yang2004}. Whilst MARL was once considered to be non-identifiable
    (there is no clear way for a third party to identify the algorithm that a particular agent was
    trained on) and black box (there is no clear way of ensuring a desired result from a learning
    algorithm from its behaviour during and after learning), a steady stream of research is emerging
    to lift this fog \cite{Bloembergen2015}. This allows for researchers to choose the parameters of
    their learning algorithms appropriately to ensure that learning agents behave in the desired
    manner.

    \section{Problem Statement} \label{sec::Problem_Statement}

    Based on the successes of the swarm control and MARL communities, a natural question to ask is
    whether the promises of these disciplines may be unified. Such a combination would radically
    improve the capabilities of autonomous system. Ultimately such a system must be driven under the
    influence of control inputs to perform given tasks. Examples of this might be: forming around
    individual components of space debris and moving them to a desired location (e.g. a bin), or
    relocating to multiple areas in a disaster zone to provide immediate relief. For such tasks to
    be achieved in dynamic environments, which is typically the case, the system must be able to
    constantly adapt its approach and respond accordingly. Given these considerations, the problem
    may be stated as follows

    \begin{enumerate}
    	\item A population of robots is given a goal which must be achieved, which can often be
    	deconstructed into a set of smaller tasks, to be executed simultaneously. 
    	\item These are to be performed in a dynamic environment with obstacles present and where
    	the number of tasks, and their priorities may be in constant flux.
    	\item The agents in the system must, therefore, make decisions collectively which will result
        in the system achieving the given goals. Due to the dynamic quality of the problem, these
        decisions may need to be evaluated online and in a decentralised fashion, as it may not be
        possible for a centralised decision maker to communicate with all agents. 
        \item It is desirable, therefore, to control the dynamical behaviour of
    	a system composed of agents who learn and adapt through interactions with each other. In
    	addition, to ensure the safe operation of the system, properties such as state and control
    	constraints must be satisfied.
    \end{enumerate}


    With the goal of addressing this problem in mind, the aim of this PhD is to address the question
    posed by the final point: how does the dynamical behaviour of a swarm of intelligent, adaptive
    agents evolve under the influence of defined control inputs?

    \section{Objectives and Scope} \label{sec::Objectives_and_Scope}

    In particular, we will study

    \begin{itemize}
    	\item How the strategies of a population of agents evolves as the iteratively interact
    	with one another. We will examine whether the system evolves to a stable equilibrium, or
    	whether it exhibits complex, even chaotic behaviour.
    	\item The extent to which the dynamical behaviour of a population of learning agents is
    	influenced under given control. We seek to understand the conditions under which guarantees
    	on theoretical results such as controllability, stability, and well-posedness may be
    	established.
    \end{itemize}

    The resultant work will provide novel insights into the behaviour of swarm systems and provide
    new methods for which they may be practically used. Importantly, the study will seek to
    understand the guarantees that can be placed on the safety of such systems in terms of their
    stability and constraint satisfaction.

    % \section{Methodology} \label{sec::Methodology}



\end{document}