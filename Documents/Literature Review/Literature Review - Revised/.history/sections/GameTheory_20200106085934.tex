\documentclass[../sample.tex]{subfiles}

\begin{document}

Game Theory has a rich history when considering an understanding of multi-agent systems. These begin
in economics but have found a strong application in computation due to the rising need for
distributed systems. Game Theory, therefore, branches across all of the categories in this section
(although its synergy with swarms is requires significant development) since Dec-POMDP and MARL
methods have both used game theory to support their frameworks. In fact, Dec-POMDP is a subset of
Partially Observed Stochastic Games (POSG), in which all agents use the same payoff.

\section{Market Based Methods}

Garapati et al. \cite{Garapati2018AMissions} define a market based method as the setting where
agents "follow their own interests and establish the mechanism of a market for distributing the
tasks". Auctioning is the most widely used sub-field of market approaches and so I will use them
interchangeably.

Whilst there are different variants to auctioning, the general procedure is that an auctioneer who
has knowledge of a task (or multiple tasks) will set up an auction for said task. Agents can then
make bids on these tasks and, once the auction is complete, the highest bid will win the task. In
the specific application to robotics, a robot's bid will often reflect the costs, suitability or
utility their undertaking the task \cite{BernardineDias2006Market-basedAnalysis}. This immediately
highlights a few points. The first is that the method is not too heavily reliant upon a single
processor to determine some joint policy. Tasks are allocated on a case-by-case basis and the
utilities are calculated by the agent themselves. The only centralised process is the auctioneer's
assessment of the winner which then relays this information back to them. The downside of this is
that the system is heavily reliant upon strong communication channels, without which tasks may not
be assigned, incorrect utilities may be communicated and, in general, sub-optimal solutions reached.
 Furthermore, the requirement that the agents themselves determine the cost of their actions assumes
that they have the computational capability to do so. Furthermore, the bids placed by each agent
need to be a strong representation of their capability to perform a task which may be hard to
estimate without expert knowledge. However, market based methods are well suited to explanation
through argumentation (similar to \cite{Jung2001DistributedArgumentation}.

With well chosen payoffs, market based approaches work extremely well. For instance, in
\cite{Dias2000ASystem}, the authors show that a free market approach (where agents try to maximise
their own profits) can lead to a strong collaborative effort across teams. Similarly, in
\cite{Thomas2005Multi-robotScenarios}, Thomas et al. apply the auctioning scheme presented in
\cite{Gerkey2002Sold:Coordination} towards a robot construction team. However, it is important to
note that these are both passive settings; tasks were assigned before the team were in the field
and, in the case of \cite{Gerkey2002Sold:Coordination}, the system would repeat the bidding process
if a robot failed. While both show strong performance, it cannot be said that either would be
applicable in dangerous environments in which dynamic reassignment must happen within strict time
constraints. Stancliff et al. \cite{Stancliff2009PlanningAllocation} suggest that a more robust
method to planning would be to account for failures a priori, a philosophy which is exemplified in
\cite{Chen2010ACollaboration} who consider the robot's reliability and relevance to a task as well
as 'history relevance' which considers the relationship between pairs of robots with the aim of
producing more effective teams.

There has also been some interesting work in probabilistic verification of market based approaches. 
Most notable to me is \cite{Pallottino2007ProbabilisticAvoidance} which considers the case of
conflict avoidance. Though their method focuses on collision avoidance, it highlights the need for
verification of conflict resolution and goal achievement in market based approaches with different
payoff structures. Sirigineedi et al. \cite{Sirigineedi2010DecentralisedApproach} make a step in
this direction by considering the verification of cooperative surveillance along a route network. 
From my understanding, this means that they were able to verify that their agents were able to
traverse along the network without interference. 

\section{Stochastic Games}

\TODO{For the love of God do this already}

\section{Game Dynamics}

Game Dynamics (which I often refer to as Multi Agent Dynamics or Learning Dynamics) considers the
problem of mathematically modelling Multi Agent Systems who repeatedly interact with one another.
This model then serves to be able to predict the evolution of the system as well as to understand
the trajectory of learning. Typically, this looks at considering whether or not the method is likely
to converge towards a Nash equilibrium. This is generally a difficult problem to solve
\cite{ShohamMultiagentFoundations} for all but toy problems. To extend this applicability into real
world settings requires the study of stable equilbrium points;
\cite{Letcher2019DifferentiableMechanics} shows that the stable equilbrium and Nash equilibrium (NE)
are not necessarily the same and, in fact, argue that stable points are more informative than NEs.
Stability provides some guarantees against the stochastic nature of the environment since a stable
equilibrium will always be returned to even after perturbations. This extremely important in Safe
and Trusted AI as it provides guarantees against undesired behaviour in real world environments.

\TODO{Add in the papers from Mendeley which are useful to this part of the study}

\section{Game Theoretic Control}

Game theory can often be applied to problems of control theory (particularly where there are
multiple agents) to develop robust controllers which guarantee properties of stability and
constraint satisfaction. 

This idea is explored in \cite{Marden2018AnnualControl}. Here, a zero-sum game is considered in
which the players are a controller and an adversarial environment. The design of the controller must
be such that it is able to drive the system to zero error. To illustrate, consider the problem of
designing a controller for a re-entry vehicle, as in \cite{Breitner1994ReentryGame}, in which
vortices seek to destabilise the agent. This will allow us to build stable agents in a much more
efficient manner since we can simulate the adversarial environment and hypothetical scenarios the
agent may encounter without actually encountering them. The same notion is explored by Bardi et al
\cite{Bardi1991DifferentialDisturbances}.

Mylvaganam et al, in \cite{Mylvaganam2017AutonomousApproach}, consider the N-robot collision
avoidance problem, similarly from the point of view of differential game theory. They develop a
robust feedback system for the robots which they show to be able to drive the system towards
predefined targets whilst providing guarantees of interference from other agents (or lack thereof).
In \cite{MylvaganamASystems}, Mylvaganam also considers a game theoretic control of multi-agent
systems in a distributed manner. Here, agents only consider their own payoff structure and have
limited communication with one another. The author shows that an approximate equilibrium can be
found using algebraic methods and illustrate the capabilities of the technique using a collision
avoidance example. For the sake of brevity, I will not include all of the numerous strides that
Mylvaganam has introduced to the area. However, I must conclude with those presented in
\cite{Mylvaganam2014}. Here, the author presents approximate solutions to a number of differential
games, including linear-quadratic differential games (in which system dynamics are linear functions
whilst payoff functions are quadratic), Stackelberg differential games, where a hierarchy is 
induced across the players (a notion was suggested in the research proposal) and mean-field games,
which is discussed in 'MARL'. The importance of the linear-quadratic
differential game is the stability of the solution; solutions for the Nash equilibria are
admissable iff they are locally exponentially stable (which the author often shows with the aid of
Lyapunov functions). Approximate solutions to the NE are developed which are more feasible to
calculate online. The author then shows that this is not simply a theoretical exercise by applying
the novel methods towards multi-agent collision problems and designs dynamic control laws which
guarantee that each agent will reach their desired state whilst avoiding collision with the other
agents. Similarly, the Stackelberg game is applied to the problem of optimal monitoring by a multi
robot system. 	


\end{document}