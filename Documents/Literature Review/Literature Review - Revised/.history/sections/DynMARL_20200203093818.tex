\documentclass[../sample.tex]{subfiles}

\begin{document}
    The purpose of this chapter is to further develop the ideas presented in Chapter
    \ref{ch::Proposals}. As mentioned, there are four main areas which have been identified for
    development in this field. These all aim to better apply the study to cases which are more
    relevant for real world Reinforcement Learning problems. To this end, the following studies are
    proposed.
    
    \begin{itemize}
        \item Dynamics of large agent populations with heterogenous agents.
        \item Dynamics of Reinforcement Learning in continuous action spaces.
        \item Dynamics of Learning in 'stateful' environments.
        \item Characterisation of complex dynamics using Reinforcement Learning Algorithms
    \end{itemize}

    In the above, the term 'stateful' is borrowed from Bloembergen et al. \cite{Bloembergen2015} and
   refers to the consideration of state transitions in the game. The remainder of this chapter will
   address current research in the above topics and avenues for attempting new results.
   
   \section{Large Agent Dynamics}

    This suggestion aims to build upon the work presented by Hu et al. \cite{Hu2019} in which the
    point is made that the vast majority of the 
    
\end{document}