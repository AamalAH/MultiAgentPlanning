\documentclass[../sample.tex]{subfiles}

\begin{document}
    The purpose of this chapter is to further develop the ideas presented in Chapter
    \ref{ch::Proposals}. As mentioned, there are four main areas which have been identified for
    development in this field. These all aim to better apply the study to cases which are more
    relevant for real world Reinforcement Learning problems. To this end, the following studies are
    proposed.
    
    \begin{itemize}
        \item Dynamics of large agent populations with heterogenous agents.
        \item Dynamics of Reinforcement Learning in continuous action spaces.
        \item Dynamics of Learning in 'stateful' environments.
        \item Characterisation of complex dynamics using Reinforcement Learning Algorithms
    \end{itemize}

    In the above, the term 'stateful' is borrowed from Bloembergen et al. \cite{Bloembergen2015} and
   refers to the consideration of state transitions in the game. The remainder of this chapter will
   address current research in the above topics and avenues for attempting new results.
   
   \section{Large Agent Dynamics}

    This suggestion aims to build upon the work presented by Hu et al. \cite{Hu2019} in which the
    point is made that the vast majority of the work done in this area considers games with a finite
    (usually two) number of agents. This limits the capability of the analysis to generalise to much
    larger agent populations, such as swarms. To allieviate this concern, the authors present an
    anaysis of the learning dynamics for a large agent population (which they approximate as
    containing infinite agents) where each agent is an independent Q-Learning using Boltzmann
    exploration. The emergent dynamics is given by a Fokker-Planck equation which is numerical shown
    to be a strong approximation of the true learning dynamics. However, as the authors point out,
    this is the first attempt at considering such a problem, and relies on heavy assumptions.
    Possible extensions therefore require a consideration of:

    \begin{itemize}
        \item Asymmetric games, in which agents have different payoffs, introducing heterogeneity
        and competition into the system.
        \item Stateful Games.
        \item Heterogeneous populations, in which there are p different populations of agents.
    \end{itemize}

    To the best of my knowledge, in the particular case of $N >> 1$ agent populations, none of the
    above have yet been attempted.

    \section{Continuous Action Spaces}

    Here lies the opportunity to extend the work of Tuyls' et al. \cite{Tuyls2006AnGames} towards
    continuous action spaces. This would be particularly useful for most robotic applications, which
    operate in continuous spaces. 

    Work has commenced in this area, but is still in its infancy. Notable examples are: 
    \cite{Letcher2019DifferentiableMechanics}, in which Letcher et al. consider the dynamics of
    Differentiable Games, though their focus is on its application to gradient descent in GANs as
    opposed to reinforcement learning, and \cite{Galstyan2013}, in which the dynamics of Q-Learning,
    with a Boltzmann action selection is considered. To achieve the result, the author replaces the
    strategy vector with a probability density function (pdf) over the strategy space, which leads
    to an integro-differential equation which describes the evolution of the pdf with time. Due to
    its complexity, the dynamics cannot be analysed as easily as those in \cite{Tuyls2006AnGames},
    and so Galstyan restricts the analysis to the steady-state solution (i.e. the fixed point) of
    the dynamics. Their analysis shows strong agreement with the designed experiments, under varying
    payoff conditions (note that the payoffs are now a function, rather than a discrete matrix).

    Galstyan goes on to present avenues for further work. These are summarised below:

    \begin{itemize}
        \item  An analysis of the steady state equations, considering the existence and uniqueness
        of solutions under varying payoff structures. How do these compare with the underlying Nash
        Equilibrium. Tuyls and Westra also suggest the consideration of stability analysis in this regard.
        \item A consideration of larger agent populations. No attempt towards this has yet been made
        as far as I am aware.
        \item A consideration of these dynamics under state transitions. Tuyls and Westra suggest
        that an analysis using switching dynamics (Chapter \ref{ch::ControlTheory}) may be useful
        here, but no attempt has yet been made as far as I am aware.
    \end{itemize}

    Tuyls and Westra \cite{TuylesWestra} also consider continuous action spaces from the same
    perspective as Ruijgrok and Ruijgrok \cite{Ruijgrok}, namely that of the replicator dynamics.
    The former extend the analysis of the latter by considering the case where mutations are
    deterministic (e.g. through epsilon-greedy exploration) and these mutations only allow for small
    changes within the strategy space. From an EGT perspective, this analysis provides a strong
    characterisation of learning dynamics in continous action spaces and generalises the results of
    Galstyan to generic update rules (rather than the traditional Q-Learning approach). 

    \section{Stateful environments} \label{sec::Stateful}

    This is potentially the most relevant of the sections in regards to applicability to
    reinforcement learning as it extends the typical consideration of stateless normal-form games,
    in which the payoffs and strategies of each player is well defined, to stochastic games, in
    which games have probabilistic transitions across them. I have yet to explore this section in
    depth.

    \section{Characterisation of Complex Dynamics}

    The intention of this area of study is to consider the ideas presented in works such as 
    \cite{Sanders2018} and \cite{Galla2011} which consider complex behaviour, including cycles and
    chaos, in certain games using simple reinforcement learning algorithms. These provide a great
    deal of insight into whether or not the game will converge and, if so, to what equilibrium.
    Galla, for instance, shows that adjusting a memory parameter when learning to solve an Iterated
    Prisoner's Dilemma game can shift the equilibrium from one showing purely defective behaviour to
    one showing cycles of cooperations and defection. Sanders et al. generalise this result by
    showing that, for a particular learning algorithm known as Experience Weighted Attraction,
    learning dynamics varies dependent on the mutual effect of two parameters. 

    This determination of where RL algorithms will show complex dynamics is important, not only for
    a priori understanding of resultant behaviour, but is particularly important for algorithms in
    which agents aim to predict the behaviour of other agents. For this to be feasible, the learning
    dynamics must not exhibit chaos, otherwise it would be impossible for an agent to make any
    reasonable predictions about the future behaviour of its opponents. 

    \cite{Amato2015PlanningUncertainty}
    
\end{document}