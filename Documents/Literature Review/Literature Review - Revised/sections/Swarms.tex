\documentclass[../sample.tex]{subfiles}


\begin{document}

Swarm based systems comprise of multiple homogeneous agents which are able to organise themselves
through a formation using a series of simple local interactions with their neighbours
\cite{Couceiro2015}. Whilst the individual agents are generally simplistic, the collective behaviour
may exhibit complex phenomena emulating systems observed in biological organisations such as bee or
ant colonies \cite{Sethi2017}. Hybrid algorithms such as in \cite{Gao2018} show an accelerated
performance in reaching globally optimal solutions in search-based tasks. The advantage of many
swarm algorithms is that they are based on local interactions and so are incredibly scalable
\cite{Rizk2018}.

\section{Decision Making in Swarms}

The reduction of the complexity in interactions between agents also allows for the robots to perform
other calculations on board. In \cite{Pini2011TaskSelection}, Pini et al. leverage this by
considering adaptive task partitioning across swarms. This allows a swarm, in a decentralised
manner, to deliberate whether to partition a task into its sub-tasks or to perform the task in its
whole. As of now (to the best of my knowledge) the problem of partitioning general tasks into its N
sub-tasks is unexplored. This, however, highlights another advantage of swarm systems; they are
readily divided into sub-groups (as in \cite{Zahadat2016DivisionInhibition}) to perform a
divide-et-impera (divide and conquer) approach to solving problems \cite{Pini2011TaskSelection}). 

Furthermore, swarm systems may be designed in a leaderless manner and so do not require the use of a
central controller \cite{Couceiro2015}. This presents the advantage that the system can rapidly
adapt to the loss of agents or separation of groups throughout the task. However, the assumptions
made regarding the homogeneity of individual agents and the simplicity of their local interactions
result in significant limitations placed on the complexity of the tasks that swarm systems can
accomplish.

\section{Approaches to Swarm Control}

Swarms systems are often developed by considering their natural analogues, such as those seen in
fish, birds, bacteria, or (most commonly) ants. 

\TODO{Find typical approaches towards swarm solutions.}

Recent work has seen the advancement of swarms controlled in stochastic environments. This is
particularly important for swarm intelligence in robots; many systems are developed with the
motivation of search-and-rescue, in which the robot swarms will have to operate in environments
where accounting for uncertainty is critical. To this end, swarm systems have seen the advent of
stochastic control. In this, the swarm is modelled as a diffusive system using a stochastic
equation, most often the Kolmogorov Forward Equation \cite{Hamann2008}. In \cite{Hamann2008}, the author shows that this equation can be derived by considering local trajectories of
smaller subsets of the swarm. Elamvazhuthi and Berman \cite{Elamvazhuthi2019} extend this idea to other
stochastic models for swarms, importantly considering an advection-diffusion-reaction model which
allows for hybrid agents to switch between different modes of operation. 

The above models come under the term ‘mean-field models’ \cite{Elamvazhuthi2019b}, a series of
equations
for stochastic forward processes (such as swarm foraging) in which, as the number of agents tends to
infinity, the true macroscopic motion of the system tends to these equations. Importantly, however,
these stochastic equations allow for an analysis of the swarm, as well as the ability to develop
control laws. In \cite{Li2017}, Li et al show that these models can be used to develop control
laws for robots in a swarm and drive them towards a target distribution. The method is shown to
perform accurately both in simulation and on real robots. The advantage is that guarantees can be
placed on the convergence and stabilisabiilty of the swarm towards the desired distribution.
However, there appears to be significant scope to expand upon this methodology from a safety
perspective. To begin with, the method does not consider inter-agent interactions and therefore does
not formally guard against collisions. In \cite{Inoue2019}, a similar problem is considered,
although collisions are avoided by having the robot simply move in the opposite direction when
encountering an obstacle. It is here that the mean-field models are not as strong since they do not
take these local interactions into account. It must be noted, however, that Inoue et al have
identified this problem and are currently working towards its incorporation into the model.

It would, therefore, be of particular interest from a safety perspective to consider agent
interactions. A starting point may be \cite{Bellomo2017} which extends the dynamical model of a
swarm system to include agent interaction. This presents a first step towards considering the swarm
as composed of intelligent agents rather than mindless particles and, as the authors suggest,
presents the possibility of applying game theoretic approaches towards swarms, which also gives the
ability to consider heterogeneous swarm systems who interact with one another through repeated play.
Similarly, it would allow for a stronger control perspective on swarm systems such as presented in
\cite{Borzi2015}, in which a model predictive control (MPC) scheme is presented for a
leader-follower swarm system to achieve given tasks. 


\section{Heterogeneous Swarms and Self-Healing}

Recently, there has been an increased interest in introducing heterogeneity into the swarm systems
to improve their real world applicability. An example of this which have been shown strong real
world success can be found in \cite{DucatelleSelf-organizedSwarms}. Here, the authors consider two
swarm teams, referred to as 'foot bots' and 'eye bots' who work in unison to explore an environment
and solve a navigation task.

\TODO{Do more research into heterogeneous swarm systems to pad this out. Perhaps we could introduce
heterogeneity from the point of view of hybrid systems in the stochastic control section.}

This area of research is sparsely populated and warrants further exploration. This is since the use
of co-evolutionary teams can improve the robustness of the swarm optimisation. This is since it will
be possible for a team to automatically determine when robots in the other team are not exhibiting
expected behaviour and ensure that the other team self corrects. This process is referred to as
'Self-Healing' in \cite{LiuTrust-Aware}. Here, the authors allow a user to define the goals of a
swarm system. From this, a 'trust' metric can be defined which measures the deviation of each agent
from the expected behaviour. The self healing process occurs by limiting communication of all agents
with 'untrusted' agents and encouraging communication with 'trusted' agents. The unison of
self-healing with co-evolution may present the opportunity for heterogenous swarms to maintain their
evolutionary stability, even in the face of environmental disturbances. 

\section{Fault Detection and Recovery}


The above sections have considered the fact that swarm systems are robust to losses in the group. 
However, any MAS system must first be able to recognise that an agent has undergone some failure.

To this end, Tarapore et al. \cite{Tarapore2019FaultDetection} develop a robust fault detection
approach in which the swarm itself, in a decentralised manner, is capable of assessing deviation
from 'normal' behaviour, even when the behaviour of the swarm itself is altered (perhaps by a remote
operator). The authors achieve this by requiring that the agents themselves sense and characterise
their own behaviour. This characterisation is formulated as a binary feature vector which is then
communicated to the agent's neighbours. These neighbours will reach a consensus over whether the
agent should be treated as faulty based on their collective behaviour. The results presented in
\cite{Tarapore2019FaultDetection} show extremely promising results and suggest that their method is,
in fact, able to determine faults with high accuracy in the presence of various fault types
(including sensor and actuator failures), although poor performance is seen in actuation failures in
some instances. It should be noted that this method requires that each robot transmit their feature
vector to the nearest neighbours. In environments where communication may be severely limited, this
may present further errors. Furthermore, it is unlikely that, when a robot is damaged, only one of
its components will be affected. Therefore, it is important to determine the effect on performance
in the face of multiple agent failures and in communication losses. This exploration may open the
possibility of improving the state of the art in terms of failure detection in swarm systems. 

It may be interesting to consider failure from the point of view of stochastic control with
jump dynamics (See 'Hybrid Control'). For instance, we could consider the development of systems
which are robust to different modes of operation based on failure. Similarly an exploration into
adaptive control could allow us to develop swarm systems who can recover from discrete jumps based
on belief models.


\end{document}