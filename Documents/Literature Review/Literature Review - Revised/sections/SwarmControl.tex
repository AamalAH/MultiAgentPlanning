\documentclass[../sample.tex]{subfiles}

\begin{document}
	
	This chapter expands upon the research directions presented in Chapter \ref{ch::Proposals}.
	Elaborating on the previous discussion, we propose to consider the development of control
	methodologies for complex swarm systems. This is strongly influenced by the work presented by
	Bellomo et al. \cite{Bellomo2017} who consider swarms as large systems of living agents (termed
	'active particles'). Crucially, this accounts for agents which exhibit complex
	agent-agent interactions. This interaction can range from simple attractive/repulsive
	forces to strategy selection. To the best of my knowledge, no control methodologies have
	been proposed for this model of swarms. In detail, the authors consider the following features:

	\begin{enumerate}
		\item The swarm consists of a large number of active particles who interact with other
		particles within a sensitivity domain.
		\item Interactions are non-local and nonlinearly additive.
		\item The swarm occupies an initial domain which, through the overall dynamics, evolves in
		time.
		\item The swarm may contain different types of subgroups. For example, there may be leaders
		in the group. In a broad sense, the approach allows for groups which follow different
		strategies and interaction rules with other groups.
		\item The rules of interaction dynamics depend on the type of agents present, but is
		uniformly distributed within each group.
	\end{enumerate}

	The advantage to this view of swarming agents is that it is extremely general. In fact, to the best
	of my knowledge, it is the most general swarm description available in the literature. It accounts
	for heterogeneous groups, inter-agent interactions, which maybe both short and long-range. This is
	particularly important from a safety perspective; most existing considerations of swarms treat the
	agents as evolving through independent, random motion and, therefore, do not account for collision
	avoidance behaviour. However, to control swarms of intelligent agents who do interact with one
	another, it is important that interactions are taken into account. Similarly, this is beneficial
	from an intelligence perspective; the inter-agent interactions can also incorporate the evolution
	of agent strategies through repeated interactions (i.e. Multi Agent Reinforcement Learning). This
	notion is expanded upon in Chapter \ref{ch::DynMARL}.

	With the ultimate goal of producing control strategies for swarms of active particles, the
	following studies are proposed:

	\begin{enumerate}
		\item Control of a homogeneous swarm through field interaction,
		\item An experimental study of swarm field interaction,
		\item Control of a swarm through leadership,
		\item Generalisations of control strategies,
		\item Consideration of learning strategies through repeated interactions.
	\end{enumerate}

	The above is ordered in an indicative manner of the general progress of research with
	complexity gradually being introduced over time. However, the order is not immutable. In
	particular, point 5, the consideration of learning, may be introduced into Points 1-4 throughout the study. The remainder of this chapter addresses each of these points and elaborates on the
	suggested work.

	\section{Control through fields} % (fold)
	\label{sub:control_through_fields}
	
	This study extends the work of Zhang \cite{Zhang2018}, Elamvazuthi \cite{Elamvazhuthi2019} and
	Li et al. \cite{Li2017}. Here, the authors consider the dynamics of swarms through 'drifted
	brownian motion'. In this case, the control input is some scalar field which is generated from the
	environment. Each agent measures this field at their location and uses it to adjust their
	velocity, which eventually results in the entire swarm achieving a desired distribution. This technique
	has been shown, both theoretically and experimentally, to drive swarm systems in a stable
	manner.

	The hypothesis of this section is that the distribution of active particles may
	similarly be controlled through the influence of a scalar field. The dynamics derived by Bellomo
	et al. \cite{Bellomo2017} includes a 'flocking field' which agents can measure. By controlling
	the flocking field, we may be able to drive the swarm to specific
	distributions. 

	The contribution that this study presents over those using brownian motion is the inclusion of an
	interaction operator. This governs how agents interact with one another. By accounting for this, we
	are able to control swarms where agents, for instance, avoid collisions with one another. The
	complexity of this interaction term will need to be gradually increased over time. To begin with,
	we may entirely neglect this term. It is here that we will compare the model proposed by Zhang with
	that proposed by Bellomo et al. We may then consider only local interactions, which is typical for
	most swarming systems in the current literature, before then considering non-local interactions.
	This provides the scope for swarms to interact with a greater number of agents. At this
	stage, we may expand our study to consider the presence of interaction domains. This places
	constraints on how agents may interact with one another, allowing for a greater generalisation of
	inter-agent interactions. Ultimately, we would like to influence the interaction term, though this
	is discussed in Section \ref{sec:learning_strategies_through_iterated_interaction}. The questions
	we will consider here are:

	\begin{itemize}
		\item The controllability of the system: Under what conditions is it possible to drive the
		swarm from one configuration to another in a stable manner?
		\item The guarantees of the system: Is it possible to ensure that the agents will not enter
		certain areas of state space, or exceed certain control constraints (e.g. max imum velocity)?
	\end{itemize}

	Some applications of this study are:

	\begin{itemize}
		\item Search and Rescue, in which agents will be required to flock to specific areas (e.g.
		areas with survivors detected).
		\item Construction, in which agents will be required to take on desired formations based on
		the structure to be designed.
		\item Medical Robotics, in which agents are required to target specific areas of the body.
	\end{itemize}


	In conclusion, the results from this study already significantly expand the current capabilities of
	swarming agents by developing control methodologies which can drive the swarm towards desired
	distributions in a stable manner. This is accomplished through controlling the 'flocking field'
	which is indirectly used to adjust the velocity of the agents.

	% section control_through_fields (end)
	
	\section{Experimental Study of Fields} % (fold)
	\label{sec:experimental_study_of_fields}
	
	The natural question which arises from the previous discussion pertains to the 'flocking field' and
	what physical property this may correspond to. In other words, where does this field come from? In
	the previous section, we assumed, a priori, the existence of this field. However, this will not
	necessarily be the case in reality. We will therefore turn to experimentation to determine how we might
	generate this field for given scenarios. Examples of how the field may be generated are:

	\begin{itemize}
		\item Measurements of physical properties, such as the intensity of light or heat 
		(particularly useful when aiming to avoid fires in urban search and rescue), or
		concentration of pollutants for environmental applications.
		\item Density of the environment through computer vision. For instance, a centralised sytem may map the environment and artificially generate a field which it then communicates to the swarm.
		\item Shape detection for moving objects. For instance, picking up beams and bringing
		them to desired locations may be solved by appropriately choosing an underlying field.
	\end{itemize}

	These suggestions give us the capability to verify the results presented in the previous
	section. They will also allow us to show the applicability of the results in a wide array of
	disciplines.
	% section experimental_study_of_fields (end)

	\section{Control through leadership} % (fold)
	\label{sec:control_through_leadership}
	
	It is here that we start to consider heterogeneity within the system. To this end, we consider the
	case that the majority of the swarm are comprised of 'followers' to whom controls cannot be
	directly applied. Instead, controls are applied to a select group of 'leaders' within the swarm.
	Then, by their mutual interaction, the leaders are able to herd the swarm towards particular
	states. This study will follow a similar structure to that of Borzi and Suttida  \cite{Borzi2015}
	and Ko and Zuazua \cite{Ko2019}. In both of these works, a drawback is that no requirements are
	placed on the final velocity of the followers. This means that whilst the swarm may reach a desired
	state, it will not remain in that position. The questions that we will consider here are similar in
	nature to those presented in Section \ref{sub:control_through_fields}, namely controllability and
	guarantees but will aim to resolve the issue of final velocity.

	The disadvantage to control through leadership is that we have induced points of failure (i.e.
	if the leaders fail, we can no longer control the system). However, its advantage as compared
	with control through fields is the capability to handle dynamic environments. Before, if we were
	to have dynamic environments, we would have to adjust the flocking field accordingly and
	communicate this change to the entire swarm. Here, the followers do not require any knowledge of
	the environment as they only interact with the leaders. This reduced computation allows for a
	greater capacity for the swarm behaviour to evolve dynamically.

	% section control_through_leadership (end)

	\section{Generalisation of Control Strategies} % (fold)
	\label{sec:generalisation_of_control_strategies}
	
	This section of study will be broken down into two areas of exploration. These are:

	\begin{enumerate}
		\item Agent Heterogeneity
		\item State Space Non-linearity
	\end{enumerate}

	\paragraph{Agent Heterogeneity} % (fold)
	\label{par:agent_heterogeneity}
	
	* builds on Bellomo et al's proposal of sub-groups within the swarm. These groups may interact with
	the flocking field or the leader in different ways. For instance, one may be more strongly repulsed
	by the leader than another, or one group may experience a completely different flocking field. However, they all interact with each other through the interaction operator. In addition,
	we may consider that, through the course of the swarming process, some agents may fail, and
	therefore its internal dynamics and ability to interact with other agents will change. We must
	ensure that the control framework (which may now take on a hybrid control \cite{Lygeros2004}
	format) can account for this and still ensure the completion of the task and safety of the system.
	This will be a powerful extension to swarm literature since, to the best of my knowledge, there are
	no studies which provide a unified framework with which to control agents of different types.

	% paragraph agent_heterogeneity (end)

	\paragraph{State Space Non-Linearity} % (fold)
	\label{par:state_space_non_linearity}
	
	takes into account the mechanical constraints of the agents. This is inspired by the work of
	Elamvazuthi and Berman \cite{Elamvazuthi2018} who account for the fact that the behaviour of
	agents is best described by their configuration space, rather than simply their state space.
	For instance in order for a group of robot manipulators (robot arms) to pick up a heavy object,
	not only must they be in the correct position (e.g. surrounding the object), but also the arms
	must be in the correct configuration to perform the action. This will allow for more complex
	interactions between agents since, rather than simply achieving some desired position, the
	agents may actually be in the required configurations to perform tasks.

	% paragraph state_space_non_linearity (end)

	The above extensions aims to generalise the control strategies developed in this study. These will be
	powerful additions to swarm literature since they will allow for systems to act in more complex
	manners by introducing heterogeneity and configurations within one control framework. 

	% section generalisation_of_control_strategies (end)

	\section{Learning Strategies through Iterated Interaction} % (fold)
	\label{sec:learning_strategies_through_iterated_interaction}
	
	This section of the study is perhaps the strongest extension to swarm control proposed in this
	chapter, and will likely be the most challenging. Here, we attempt to control the interaction term
	in Bellomo et al's model and we consider that the behaviour of the entire swarm may be influenced
	by decision making on an individual level. This will involve agents learning their strategies
	through repeated interaction. The goal, then, is to design the agent parameters, reward structure
	and control strategy to achieve desired results. To accomplish this will require a unification of
	multi agent learning and control theory. Whilst, at first glance, such an endeavour seems
	incredibly challenging, our work in the predictability of mean field learning (Chapter 
	\ref{ch::DynMARL}) places us in a unique position to tackle the problem. This framework will
	first establish the desired strategy to be taken by the agents and choose the game's reward
	structure so that learning will converge to a stable fixed point. Then by accounting for this
	strategy within the interaction term, we apply the appropriate controls in the flocking field to
	drive the swarm in a stable manner. We will again consider the aforementioned controllability
	and guarantees questions and establish how the incorporation of learning affects these results.

	% section learning_strategies_through_iterated_interaction (end)

\end{document}