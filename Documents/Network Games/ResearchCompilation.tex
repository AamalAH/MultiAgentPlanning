\documentclass{article}
\usepackage[utf8]{inputenc}
\usepackage{geometry}
\geometry{margin=2cm}
\usepackage{xcolor}

\newcommand{\ah}[1]{\textcolor{blue}{#1}}
\newcommand{\fb}[1]{\textcolor{red}{FB: #1}}

\title{Controlled Swarms through Learning on Network Games: Research Compilation}
\author{Aamal Hussain}
\date{}

\begin{document}
    \maketitle

The goal of this revision is to re-evaluate the assumptions and
guarantees placed on the following problem: to what extent is it
possible to control the behaviour of a large group of agents subject
to partial information. In particular, we seek to provide theoretical
guarantees concerning the behaviour of these agents from the point of
view of safety. It is, therefore, required that we ensure that the
tools we are using are suited to such an analysis. In this spirit,
this document outlines an overview of the research into the
game-theoretic and learning approaches towards learning for large
systems of agents.


\subsection*{Network-aggregative games are a stronger model for swarm systems than mean-field}

In particular, I focus this research on the problem of
network-aggregative games \cite{Parise}. In this class of game, we
consider an aggregative (large population) system which has an
underlying communication network, i.e. agents adapt their behaviour
dependent on the behaviours of their neighbours in the network, but
their knowledge of the other agents is limited (or even non-existent),
This form of game appears to be better suited to the problem of swarms
than the mean-field model, in which an agent would be required to have
knowledge of the average behaviour of the entire population
\cite{Paccagnan2019}. This assumption is infeasible for a fully
distributed system, unless the agent is somehow able to communicate
with the entire population and perform the necessary calculations to
find the average behaviour of the population.


In addition, the research on learning on network games (sometimes referred to as graphical games
\cite{Li2017}) is a topic which is increasingly being turned to from the
learning and game-theoretic communities alike. The major classes of learning algorithms which have
been explored in this context are: Fictitious Play \cite{Ewerhart2019}, Q-Learning 
\cite{Vamvoudakis2017} and No-Regret \cite{Nagarajan2018}. All
of these continue to undergo rigorous theoretical analysis from the game theoretic and computational
learning communities. 

In this spirit, important facts about learning on networks have been
established for each of the learning algorithms. This includes, most
importantly, convergence to Nash Equilibria. For each of these classes
of games, the conditions required for this behaviour are established
in: \cite{Ewerhart2019} for Fictitious Play, \cite{Nagarajan2018} for
no-regret, and \cite{Li2017} for Reinforcement Learning. For the last
of these, however, it is important to note that, whilst the paper
presents an algorithm for finding the NE, the focus of the study
concerns forming control policies based on the NE. This is not a
disadvantage of the paper but rather an advantage: namely that the
learning community tends to be more invested in the application of the
algorithms towards real-world problems than the game-theoretic
community.

The existence of these results means that there is a basis towards
forming safe learning systems on network games, since we can study the
convergence towards an equilibrium. In addition, this class of game
can, particularly for Fictitious Play and Q-Learning, be extended
towards network-aggregative games, thereby allowing its study for
larger populations. Such a study can be found in \cite{Parise}, in
which the authors produce two NE seeking algorithms which, although
they do not call it so, is remarkably similar to Fictitious Play. This
provides a useful (and perhaps only) study on network-aggregative
games which can be used to compare future work against.

\fb{idea: an analysis as in our aamas paper, but grounded on network games?}

\subsection*{The work on learning on network games is quite recent.} 

To qualify the above statements, it is important to note that the work
in learning on network-games and, in particular, network-aggregative
games is quite new. Indeed the publications listed in this review,
which represent some of the major breakthroughs in network games, all
fall within the span of 2015-2020. Whilst these, of course, build on
previous work, it is important to note that the specific problem of
learning and control on network-aggregative games is an open
problem. This is because network games can be considered to be one of
the more generalised forms of game since it imposes an underlying
communication network across the system. Indeed, a game in which an
agent acts according to all of the other agents in the group can
simply be considered a fully-connected network. There are, however,
still further generalisations that could be made. For instance,
learning on weighted networks has received some attention
\cite{Bayer2019} but still requires maturity. Furthermore, directed,
or asymmetric networks and time-varying networks are topics left, as
of yet, unexplored. This exposition is not to suggest directions for
our own research, but merely to present the fact that the problem of
learning on network games is a valuable one to consider due to its
generality.


\subsection*{Then comes the question of which learning algorithm to study.} % (fold)

Hopefully, I have made a strong argument for why network-aggregative
games are a good class of games to focus the study on for the problem
of learning swarm behaviour. It then becomes incumbent on us to choose
a class of algorithm to consider. Whilst each algorithm presents its
own set of pros and cons, I argue that choosing the algorithm is
merely a matter of considering the assumptions that would be realistic
to place on our system. As you mentioned in our discussion, if it is
the case that the problem we consider satisfies the assumptions for
one of these classes, it becomes necessary that we choose this class
to study.

I would argue that the major conditions that we need to place on our
system is that the agents have no knowledge of the behaviour of the
other agents and must, instead, infer that behaviour. Swarm systems
are built on the assumption that an agent is able to interact with,
and observe, only a few agents in the population. This would typically
include its nearest neighbours and, if the situation demands, a leader
or set of leaders.

The other typical assumption placed on learning systems is that the
agent does not have access to the utlity function, from which it
derives its behaviours. Such a problem would occur in games which need
to be repeated over time (e.g. traffic routing problem) or for the
design of a system which is aimed to be generalisable. However, it is
unlikely that, for the problem of controlling a swarm, this assumption
would be required in general. For the most part, a control problem is
designed for a specific task in mind (e.g. coverage or foraging) and
so the objective functional is fixed.  Furthermore, as long as complex
gradients or Hessian inverses do not need to be calculated, it can be
assumed that the agent is capable of carrying out the necessary
calculations in order to perform its task. Otherwise, it would be
required that a centralised controller, who has access to the cost
function and can carry out the optimisation, would need to communicate
with each agent to identify its behaviour and deliver its next
action. Whilst there is nothing inherently wrong with this method,
such a system can no longer be called `distributed'.

With these in mind, then, I suggest that the Fictitious Play (FP) and
(perhaps less so) Reinforcement Learning (RL) algorithms would be
appropriate classes of games to study. This is because no-regret
learning considers that the agent is aware of the actions of his/her
neighbours and therefore, does not fulful our requirements. I argue
that FP is a stronger class for this problem description since it is
assumed that the agent knows her objective function and explicitly
tries to learn the behaviour of her neighbours. On the other hand, RL
does not require knowledge of neighbours or of the objective function
but implicitly attempts to learn both of these by evaluating the
actions. Whilst Q-Learning is more general, then, it lacks the
explicit modelling aspect that FP offers and, as such, its analytical
basis is not as strong. I propose, therefore, that a study of FP on
network-aggregative games provides the strongest model of swarm
systems. I am not, however, ruling out the option of looking at both
of these algorithms and comparing their performance and guarantees.

\fb{the above seems reasonable and well-motivated to me.}

\subsection*{Conclusions} 

In summary, I propose that, for the problem that we are considering, namely the problem of
controlling swarms of intelligent active particles, the strongest model to study would be that of
Fictitious Play (and/or RL) on network-aggregative games. This suggestion is due partly on the
similarity between the assumptions placed on this class of games and that of swarm systems as well
as the growing attention that this class of games has received from an analytic perspective. I
believe that the best way to proceed would be to extend the work of \cite{Ewerhart2019} with the
assumptions made in \cite{Parise}. A point which I would like to discuss with you is whether it would
be to attempt a similar extension of the work in \cite{Li2017} or \cite{Vamvoudakis2017}
(i.e. that of RL) and provide a comparison of the results. Regardless, both of these would lend
themselves well to considering the problem of distributed control as shown in 
\cite{Vamvoudakis2017} and \cite{Bianchi, DePersis2020}. Both of these allow for us to
consider the physical side of the problem by imposing that the learning algorithm must respect the
physical dynamics of the agents involved. 

I believe this to be a stronger suggestion towards the problem of Active Particle Control than that
which was suggested in the ESA as it allows for us to directly build on our current work with only a
minor modification directing our assumptions closer to the ones required by our problem. It also
allows us to bypass the section of controlling a swarm through PDEs and instead narrows our focus on
game-theoretic learning methods. 

\bibliographystyle{IEEEtran} 

\bibliography{IEEEfull,references}

\end{document}
