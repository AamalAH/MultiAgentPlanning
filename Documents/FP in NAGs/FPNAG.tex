\documentclass{article}
\usepackage[margin=2cm]{geometry}
\usepackage{amsmath, amsfonts, amsthm}
\usepackage{xcolor}

\theoremstyle{definition}
\newtheorem*{definition}{Definition}
\newtheorem{theorem}{Theorem}
\newtheorem{assumption}{Assumption}

\newcommand{\ah}[1]{\textcolor{blue}{AH: \textit{#1}}}
\newcommand{\svs}[1]{\textcolor{red}{FB: \textit{#1}}}

\title{Fictitious Play in Network Aggregative Games}
\author{AH, FB}

\begin{document}
	
	\maketitle
	
	The idea is to require that each agent $i$ play a best response to the history of the aggregate strategy in their neighbourhood. We can think of this as the agent being able to `measure' the aggregate strategy of its neighbours or as some centralised observer giving this reference to the agent.
	
	To formalise this, we assume that we have a set of agents $V = \{1, ..., p \}$ which can be thought of as nodes on a graph $\{V, E, W\}$ where $E$ is the edge-set (i.e. the set of pairs $(i, j)$ such that i and j are connected) and $W$ is a weighted adjacency matrix $[W]_{ij} = w_{ij}$. Let $N_i := \{j \in V : (i, j) \in E\}$ be the set of player $i's$ neighbours. Note $w_{ij} = 0$ if $(i, j) \not\in E$, $w_{ij} \in (0, 1]$ if $(i, j) \in E$. We impose the normalisation condition that $\sum_{j \in N_i} w_{ij} = 1$. This requirement ensures that the aggregate strategy $\sum_{j \in N_i} w_{ij} x_j$ (where $x$ is a probability vector) remains a probability vector. For example, in the 3 player case:
	
	\begin{equation}
	\sigma_1 := w_{12} y + w_{13} z = 
	w_{12}
	\begin{bmatrix}
	y_1 \\ y_2 \\ 1 - y_1 - y_2
	\end{bmatrix}		
	+ 	w_{13}
	\begin{bmatrix}
	z_1 \\ z_2 \\ 1 - z_1 - z_2
	\end{bmatrix}	= 
	\begin{bmatrix}
	w_{12} y_1 + w_{13} z_1 \\ w_{12} y_2 + w_{13} z_2 \\ 1 - (w_{12} y_1 + w_{13} z_1) - (w_{12} y_2 + w_{13} z_2)
	\end{bmatrix}	
	\end{equation}
	
	In the continuous time case, we're also going to assume that each agent $i$ chooses a mixed strategy $x_i(t) \in \Delta_i$ at time $t$ from its strategy space $\Delta_i$ (which is an $N-$simplex). We assume that $i$ has just the one strategy that it can play against all of its neighbours.
	
	Once the strategy is chosen, the agent plays the game and receives payoff $u_i (x_i, \sigma_i)$, where $\sigma_i$ is the aggregate strategy vector that $i$ perceives. The best response map $BR_i$ maps any $\sigma_i$ to the set $\arg\max_{x \in \Delta_i} u_i (x, \sigma_i)$. Then $x_i$ is a `best response' to $\sigma_i$ if $x_i \in BR_i(\sigma_i)$.
	
	\begin{definition}(NE)
		The set of vectors $\{ x_i^*\}_{i = 1}^p$ is an NE if, for all $i$,
		
		\begin{equation*}
		x_i^* \in BR_i( w_{ii} x_i^* + \sum_{j \in N_i} w_{ij} x_j^*).
		\end{equation*}
		
	\end{definition}
	
	\subsection*{Case: Aggregation is dependent only on its neighbours}
	
	We start with the following assumption
	
	\begin{assumption}
	$w_{ii} = 0$ for all $i \in \{ 1, ..., N \}$
	\end{assumption}
	
	Under this assumption the $BR$ map becomes
	
	\begin{equation}
		BR_i(\sum_{j \in N_i} w_{ij} x_j) = \arg\max_{x \in \Delta_i} u_i(x_i,\sum_{j \in N_i} w_{ij} x_j)
	\end{equation}
	
	Let us now make the assumption that the payoff function is bilinear and can be written as $u_i(x_i, \sigma_i) = x_i \cdot A^i \sigma_i$. Then 
	
	\begin{equation*}
		u_i(x_i, \sigma_i) = x_i \cdot A^i \sum_{j \in N_i} w_{ij} x_j = \sum_{j \in N_i} x_i \cdot A^{ij} x_j =: \sum_{j \in N_i} u_{ij}(x_i, x_j) 
	\end{equation*}
	
	where $A^{ij} := w_{ij} A^i$ and $u_{ij}$ is a bilinear payoff between $i$ and $j$ as defined by Ewerhart. With this transformation, we can move from a network aggregative system to a network game formulation (as in Ewerhart) in which each agent $i$ plays a game $G_{ij}$ against $j$ with payoff $A^{ij}$ and with the added requirement that each agent plays the same strategy against each of its neighbours.
	
	The Network Aggregative Fictitious Play requires that the agent plays a best response to the time average of $\sigma_i$. So let us have the following definitions:
	
	\begin{align}
		\alpha_{\sigma_i}(t) = \frac{1}{t} \int_{0}^{t} \sigma_i(t') \, dt'\nonumber \\
		\alpha_i(t) = \frac{1}{t} \int_{0}^{t} x_i(t') \, dt', \nonumber
	\end{align}
	
	where $x_i(t')$ denotes the action played by $i$ at time $t'$. Now we notice
	
	\begin{align}
		\alpha_{\sigma_i}(t) = \frac{1}{t} \int_{0}^{t} \sigma_i(t') \, dt' = \frac{1}{t} \int_{0}^t \sum_{j \in N_i} w_{ij} x_j(t') \, dt' = \sum_{j \in N_i} w_{ij} \frac{1}{t} \int_{0}^t x_j(t') \, dt' = \sum_{j \in N_i} w_{ij} \alpha_j(t) \nonumber \\
	\end{align}
	
	So, then, a Continuous Time Fictitious Play (CTFP) is a map $m$ such that each of its components $m_i: [0, \infty) \rightarrow \Delta_i$ satisfies $m_i(t) \in BR_i(\sum_{j \in N_i} w_{ij} \alpha_j(t))$ for all $t \geq 1$. Using the transformation from before, this is the requirement that, in the equivalent network game, $m_i(t) \in \arg \max_{x \in \Delta_i} \sum_{j \in N_i} u_{ij}(x_i, \alpha_j)$ for all $t \geq 1$. Now, Ewerhart showed that, in a zero-sum network (where the sum of all $u_i$ is zero), such an $m$ exists and any such $m$ converges to an NE.
	
	To clarify what the convergence property means in this context, each $m$ is associated with an $\alpha(t; m) = \frac{1}{t} \int_{0}^t m_i(t') \, dt'$. Let $A(m)$ denote the set of accumulation points of $\alpha(t; m)$ (i.e. the set of $\mu$ where there exists a sequence $\{t_q\}_{q = 0}^{\infty}$ such that $\lim_{q \rightarrow \infty} t_q = \infty$ and $\lim_{q \rightarrow \infty} \alpha(t_q; m) = \mu$). Convergence to an NE means that $A(m)$ is contained within the set of NE of the game. 
	
	In any case, the fact that $m$ converges to an NE for the equivalent network game means that it converges for the NAG also. \ah{To make this precise, I'll follow through the proof of Ewerhart}
	
	\begin{theorem}
		For any zero-sum Network Aggregative Game which follows the above assumption, any CTFP $m$ has the property that $A(m)$ is contained within the set of NE of the game.
	\end{theorem}
	
	\begin{proof}
		Take the Lyapunov function
		\begin{align}
			L(\mu) & = \sum_i \max_{x \in \Delta_i} \{u_i(x, \sum_{j \in N_i} w_{ij} \mu_j) - u_i(\mu_i, \sum_{j \in N_i} w_{ij} \mu_j) \} \nonumber \\
			& = \sum_i \max_{x \in \Delta_i} \{u_i(x, \sum_{j \in N_i} w_{ij} \mu_j)\} - \sum_i u_i(\mu_i, \sum_{j \in N_i} w_{ij} \mu_j) \nonumber \\
			& = \sum_i \max_{x \in \Delta_i} \{u_i(x, \sum_{j \in N_i} w_{ij} \mu_j)\} \nonumber
		\end{align}
		
		where the last equality holds because we made the assumption of a zero sum game. Making the usual transformation, we consider that $i$ is now involved in a network game in which it plays the same strategy against each of its neighbours $j$ giving
		
		\begin{align}
			L(\mu) = \sum_i \max_{x \in \Delta_i} \sum_{j \in N_i}  u_{ij}(x, \mu_j) \nonumber \\
		\end{align}
		
		Now we take any CTFP $m$ and recall that it is found by playing the best response to $\alpha(t; m)$, which of course means that it maximises $\sum_{j \in N_i} u_{ij}$. Therefore
		
		\begin{align}
			L(\alpha(t)) & = \sum_i \sum_{j \in N_i} u_{ij}(m_i(t), \alpha_j(t)) \nonumber \\
			& = \sum_i \sum_{j \in N_i} m_i(t) A^{ij} \alpha_j(t)
		\end{align}
		 
		 Now if we were to follow on using Ewerhart's proof, we would find that any accumulation point of $\alpha(t; m)$ is an NE of the network game. I.e. if $\mu^*$ is this limit point, then it satisfies that, for any $x \in \Delta_i$
		 
		 \begin{align}
		 	\sum_{j \in N_i} u_{ij}(x, \mu_j^*) \leq 	\sum_{j \in N_i} u_{ij}(\mu_i^*, \mu_j^*) \nonumber
		 \end{align}
		 
		 Now if we reverse the transformation this means that, for the network aggregative game we have the condition that for any $x \in \Delta_i$
		 
		 \begin{align}
		 	u_{i}(x, \sum_{j \in N_i} w_{ij} \mu_j^*) \leq u_{i}(\mu_i^*, \sum_{j \in N_i} w_{ij} \mu_j^*)
		 \end{align}
		 
	\end{proof}
	\newpage
	
	\subsection*{Second Case}
	
	\begin{align}
		x_i^* &\in \arg\max_{x \in \Delta_i} u_i(x_i, w_{ii} x_i + \sum_{j \in N_i} w_{ij} x_j^*) \nonumber\\
		&= \arg\max_{x \in \Delta_i} \bar{u}_i(x_i, \sum_{j \in N_i} w_{ij} x_j^*)
	\end{align}
	
	Since $\sigma_i$ shares the same strategy space as $x_i$, and we assume that this space is finite, then a NE exists. $\ah{If we decide to lift the finite strategy space requirement, we'll have to show the existence of an NE.}$ 
	
	Now, for fictitious play, we require that the agent plays against the time average of this reference $\sigma_i$. We can write this as
	
	\begin{equation}
		x_i(t) \in BR_i \left( \frac{1}{t} \int_{0}^{t} \sigma_i(t') dt' \right).
	\end{equation}
	
	By the definition of Continuous Time Fictitious Play (CTFP) given by Ewerhart, we require that $m: [0, \infty] \rightarrow \times_i \Delta_i$ is a measurable mapping such that, for each $i$, $m_i(t) \in BR_i \left( \frac{1}{t} \int_{0}^{t} \sigma_i(t') dt' \right)$. Now notice,
	
	\begin{equation*}
		x_i \in BR_i \left( \frac{1}{t} \int_{0}^{t} \sigma_i(t') dt' \right) \iff  \in x_i\in BR_i \left( \frac{1}{t} \int_{0}^{t} [w_{ii} m_i(t') + \sum_{j \in N_i} w_{ij} m_j(t')] \, dt' \right)
	\end{equation*}
	
	Let us assume that $u_i$ takes the form $x \cdot A_i \sigma_i$ where $A_i$ is the payoff matrix associated with agent $i$. Then,
	
	\begin{align}
		& x_i \in \arg\max_{x \in \Delta_i} u_i(x,\frac{1}{t} \int_{0}^{t} [w_{ii} m_i(t') + \sum_{j \in N_i} w_{ij} m_j(t')] \, dt') \nonumber \\
		\iff & x_i \in \arg\max_{x \in \Delta_i} x \cdot A_i \left(\frac{1}{t} \int_{0}^{t} [w_{ii} m_i(t') + \sum_{j \in N_i} w_{ij} m_j(t')] \, dt' \right) \nonumber \\
		\iff & x_i \in \arg \max_{x \in \Delta_i} x \cdot (w_{ii} A_i) \left( \frac{1}{t} \int_{0}^{t} m_i(t')\right) + \sum_{j \in N_i} x \cdot (w_{ij} A_i) \left( \frac{1}{t} \int_{0}^{t} m_j(t')\right) \nonumber \\
		\iff & x_i \in \arg \max_{x \in \Delta_i} x \cdot A_{ii} \alpha_i(t; m) + \sum_{j \in N_i} x \cdot A_{ij} \alpha_j(t; m).
	\end{align}
	
	where each $A_{ij} = w_{ij} A_i$ and $\alpha_i(t; m) =\frac{1}{t} \int_{0}^{t} m_i(t') dt'$ as defined by Ewerhart. We can, therefore, think of the network aggregative game as a network game in which each agent plays the same strategy against each of its neighbours, itself included. As such, any $m$ which satisfies the CTFP property for the equivalent network game also satisfies the CTFP requirement for the network aggregative game. 
	
	Continuing with this approach, let us look at the case where $\sum_i A_i = \textbf{0}_{n\times n}$ (i.e. a zero sum game). This means that $\sum_i u_i = 0$. Let $\mu_i^* = \lim_{t \rightarrow \infty} \alpha_i(t; m)$. More specifically, $\mu_i^*$ belongs to the set of accumulation points of $\alpha_i(\cdot; m)$ (which is set valued since the $BR_i$ map is set valued). To say that the game has `converged' we require that every $\mu^* = (\mu_1^*, ... \mu_p^*)$ is an NE. \ah{I'll follow through the proof of Ewerhart for the sake of completeness: }
	
	Let us take the Lyapunov function
	
	\begin{equation}
		L(\mu) = \sum_i \max_{e_i \in \Delta_i} \{u_i(e_i, w_{ii} \mu_i + \sum_{j \in N_i} w_{ij} \mu_j) - u_i(\mu_i, w_{ii} \mu_i + \sum_{j \in N_i} w_{ij} \mu_j) \}
	\end{equation}
	
	Using the same transformation as before:
	
	\begin{align}
		 u_i(e_i, w_{ii} \mu_i + \sum_{j \in N_i} w_{ij} \mu_j) & =  e_i \cdot A_i (w_{ii} \mu_i + \sum_{j \in N_i} w_{ij} \mu_j) \nonumber \\
		 & =  e_i \cdot (w_{ii} A_i) \mu_i + \sum_{j \in N_i} e_i \cdot (w_{ij} A_i) \mu_j \nonumber \\
		  & =  \sum_{j \in N_i \cup \{i\}} e_i \cdot A_{ij} \mu_j \nonumber 
	\end{align}
	
	Then
	
	\begin{align}
	L(\mu) &= \sum_i \max_{e_i \in \Delta_i}\sum_{j \in N_i \cup \{i\}} e_i \cdot A_{ij} \mu_j  - u_i(\mu_i, w_{ii} \mu_i + \sum_{j \in N_i} w_{ij} \mu_j) \} \nonumber \\
	&= \sum_i \max_{e_i \in \Delta_i} \{\sum_{j \in N_i \cup \{i\}} e_i \cdot A_{ij} \mu_j  \} - \sum_i  u_i(\mu_i, w_{ii} \mu_i + \sum_{j \in N_i} w_{ij} \mu_j) \nonumber \\
	&= \sum_i \max_{e_i \in \Delta_i} \{\sum_{j \in N_i \cup \{i\}} e_i \cdot A_{ij} \mu_j  \} \nonumber 
	\end{align}
	
	where the last equality holds because $\sum_i u_i = 0$. Again, if we were to relate this to a network game, in which each agent plays the same strategy against all its neighbours (itself included), then with the definition $ \tilde{u}_i (e_i, \mu_{-i}) := \sum_{j \in N_i \cup \{i\}} e_i \cdot A_{ij} \mu_j$, we still satisfy the zero sum condition for $\tilde{u}$. Now, we take a CTFP $m$ according to Ewerhart, with the condition that each agent plays the same strategy across all of its neighbours. This was shown by Ewerhart to converge. I.e. for any $\mu^*$ in the accumulation points of $\alpha(\cdot; m)$
	
	\begin{equation*}
		\tilde{u}_i(e_i, \mu_{-i}^*) \leq \tilde{u}_i(\mu_i^*, \mu_{-i}^*)
	\end{equation*}
	
	which, as we know, means that $\mu^*$ satisfies
	
	\begin{equation*}
		u_i(e_i, w_{ii} \mu_i^* + \sum_{j \in N_i} w_{ij} \mu_j^*) \leq  u_i(\mu_i^*, w_{ii} \mu_i^* + \sum_{j \in N_i} w_{ij} \mu_j^*)
	\end{equation*}
	
	
\end{document}