\documentclass{article}

% if you need to pass options to natbib, use, e.g.:
%     \PassOptionsToPackage{numbers, compress}{natbib}
% before loading neurips_2021

% ready for submission
\usepackage{neurips_2021}

% to compile a preprint version, e.g., for submission to arXiv, add add the
% [preprint] option:
%     \usepackage[preprint]{neurips_2021}

% to compile a camera-ready version, add the [final] option, e.g.:
%     \usepackage[final]{neurips_2021}

% to avoid loading the natbib package, add option nonatbib:
%    \usepackage[nonatbib]{neurips_2021}

\usepackage[utf8]{inputenc} % allow utf-8 input
\usepackage[T1]{fontenc}    % use 8-bit T1 fonts
\usepackage{hyperref}       % hyperlinks
\usepackage{url}            % simple URL typesetting
\usepackage{booktabs}       % professional-quality tables
\usepackage{amsfonts}       % blackboard math symbols
\usepackage{nicefrac}       % compact symbols for 1/2, etc.
\usepackage{microtype}      % microtypography
\usepackage{xcolor}         % colors



\usepackage{amsmath, amsfonts, amsthm}
\usepackage{enumerate}
\usepackage{graphicx}
\usepackage{subcaption}

\theoremstyle{definition}
\newtheorem{definition}{Definition}
\newtheorem{theorem}{Theorem}
\newtheorem{corollary}{Corollary}
\newtheorem{assumption}{Assumption}
\newtheorem*{remark}{Remark}

\newcommand{\ah}[1]{\textcolor{blue}{AH: \textit{#1}}}
\newcommand{\fb}[1]{\textcolor{red}{FB: \textit{#1}}}

\newcommand{\agentset}{\mathcal{N}}
\newcommand{\edgeset}{\mathcal{E}}
\newcommand{\weightset}{W}

\newcommand{\actionset}[1]{S^{#1}}
\newcommand{\utility}[1]{u^{#1}}

\newcommand{\wmunu}{w^{\mu \nu}}
\newcommand{\xmu}{x^{\mu}}
\newcommand{\xnu}{x^{\nu}}
\newcommand{\refmu}{\sigma^{\mu}}
\newcommand{\avgref}[1]{\alpha_\sigma^{#1}}
\newcommand{\NE}[1]{\bar{x}^{#1}}
\newcommand{\weightedsum}{ \sum_{\nu \in N^\mu} \wmunu \xnu}
\newcommand{\xnotmu}{x^{-\mu}}
\newcommand{\xmuaction}[1]{x^{\mu}_{#1}}

\newcommand{\pure}[2]{e^{#1}_{#2}}


\title{Fictitious Play in Network Aggregative Games}

% The \author macro works with any number of authors. There are two commands
% used to separate the names and addresses of multiple authors: \And and \AND.
%
% Using \And between authors leaves it to LaTeX to determine where to break the
% lines. Using \AND forces a line break at that point. So, if LaTeX puts 3 of 4
% authors names on the first line, and the last on the second line, try using
% \AND instead of \And before the third author name.

\author{%
 Aamal Hussain  \\
  Department of Computing\\
  Imperial College London\\
  South Kensington, London \\
  \texttt{aamal.hussain15@imperial.ac.uk} \\
  % examples of more authors
  \And
  Francesco Belardinelli
  % Coauthor \\
  % Affiliation \\
  % Address \\
  % \texttt{email} \\
  % \AND
  % Coauthor \\
  % Affiliation \\
  % Address \\
  % \texttt{email} \\
  % \And
  % Coauthor \\
  % Affiliation \\
  % Address \\
  % \texttt{email} \\
  % \And
  % Coauthor \\
  % Affiliation \\
  % Address \\
  % \texttt{email} \\
}

\begin{document}

\maketitle

\begin{abstract}
  The abstract paragraph should be indented \nicefrac{1}{2}~inch (3~picas) on
  both the left- and right-hand margins. Use 10~point type, with a vertical
  spacing (leading) of 11~points.  The word \textbf{Abstract} must be centered,
  bold, and in point size 12. Two line spaces precede the abstract. The abstract
  must be limited to one paragraph.
\end{abstract}

\section{Introduction}

\fb{multi agent $\to$ multi-agent}

Multi-Agent Learning requires a number of agents to adapt in response
to the behaviour of the other agents as well as the environment
\cite{Schwartz}. This feature leads to a fundamentally non-stationary
problem, which presents a challenge to designing effective learning
policies. Even for a small number of agents in the game,
learning has been shown to lead to non-stationary, even chaotic
behaviour \cite{SatoChaos}, and this problem becomes even more
pronounced as the number of agents increase \cite{Sanders}. In this
light, it would seem that, when there is a large population of agents,
it is almost impossible to understand the long term behaviour of multi-agent learning \cite{PiliourasChaoticMaps}. This, of course, poses a
problem for AI safety, for which it is \fb{desirable} that learning systems
are designed so that their ultimate behaviour is guaranteed to satisfy
some predetermined goals.

\fb{if space allows, you might consider to contrast the situation with
  single-agent learning.}

A promising approach to solve multi-agent learning
%To deal with this problem, one must attempt
consists in reducing the many player game to something which is more
tractable. A number of reduction methods have been proposed in recent
years, most notably \emph{mean field} games \cite{} and \emph{aggregative}
games \cite{}. The former make the assumption of an infinite number of
agents, so that the population can be represented through a
distribution over players' states \cite{CainesPaper} . Every agent
then updates their action profiles depending on this distribution. In
the latter each agent plays \fb{?} considers a real valued function which is
a convex combination of the states of the other agents. In both
formulations, a many-player game is reduced to the set of two-player
games which each agent is involved in \fb{?}.

Both approaches have been the object of rigorous study and it has been
shown that agents reach an equilibrium when learning on such games
(cf.~\cite{MFGLearningPapers} for mean field games and
\cite{Aggregative Papers} for aggregative games). However, both also
present a fundamental limitation.  Namely, they both require that
agents have access to the action profiles of the entire population.
This could be through communication with all other agents, or through
the intervention of a central coordinator who is able to \fb{?} the
entire population. Whilst recent work aims to relax this assumption
through the introduction of noise \cite{MFG-FP} or partial
observability \cite{AAMASPaper}, the requirement that each agent
updates their actions based on the entire population is rather strong
\fb{and not always supported by empirical evidence}.

In this study, we investigate a variant of aggregative games:
%known as the
\emph{network aggregative} (NA) games. This format assumes that
there is an underlying communication network through which agents
interact. Then each agent updates their actions according only to
those agents with whom they communicate. This assumption significantly relaxes
the communication load on each agent and lifts the need for a central
coordinator. Recent work on NA games has shown that it is possible
for agents to reach an equilibrium strategy in an entirely distributed
manner \cite{Grammatico, LeaderFollower, MyopicAgents}. We contribute
in this direction by analysing the long term behaviour of multi-agent learning on
 NA games. In particular, we analyse the fictitious play learning
algorithm \cite{Brown, Harris}, in which agents are assumed to be
myopic, in that they react solely to the previous behaviour of the
others.

\fb{maybe above we can reduce a bit the discussion of NA and motivate
  a bit more FP.}


\subsection{Contributions}

  The main contribution of this work is to introduce learning in
  network aggregative game through the action of fictitious play (FP).

  We first show that \fb{in aggregative games played with FP} a Nash Equilibrium exists and that FP admits
  solutions in this setting. Specifically, we consider zero-sum games
  and show that FP converges to a fixed point, which for networks
  without self-loops corresponds to a Nash equilibrium. In
  addition, we find that, for games which are not zero-sum, agents
  following FP are able to achieve no regret.

  Further, we explore FP through numerical simulations to check
%onsider the question of
whether it converges in non-zero sum games. We answer this in the
negative by finding a family of games in which action profiles cycle
around the Nash equilibrium. Finally, our experiments document how
noise affects the convergence of FP, suggesting
%Our experiments suggest
that,
under the presence of noise, the algorithm still reaches a fixed
point, but perhaps not the Nash equilibrium. This presents an
interesting avenue for future research.

\subsection{Related Work}

\paragraph{Network Aggregative Games}

NA games are a recent \cite{Parise2015} extension of
%introduction  which presents
%a variant of
aggregative games by adding an underlying structure to
the agent population. Since its introduction, distributed algorithms have
been built with the aim of finding NE in NA games. In particular,
\cite{Parise2015,Parise2020} consider the case in which payoffs are
given by Lipschitz functions with unique minimisers and apply standard
topological fixed-point towards designing algorithms that
converge to NE. We note that the \emph{Mann Iteration}, one of the
algorithms applied in \cite{Parise2020}, shares a remarkably similar
form with the discrete variant of FP. To the best of our knowledge,
this link has not been explored in the literature and is a venue for
research that we are currently pursuing. Another common algorithm for
distributed NE seeking is the projected gradient (resp. subgradient)
dynamics, which is explored in \cite{Zhang2020}
(resp.~\cite{Shokri2020, Shokri2021}). In all of these works, the cost
function is assumed to be convex, and therefore have a unique
minimiser. In fact this is a common assumption in works concerning NA
Games \cite{Zhu2021, Lei2020}, which we believe is due to its ubiquity
in control settings. We have not yet come across works which consider
NA games from the point of view of payoff matrices, which are more
common in multi-agent learning settings.  Furthermore, to the best of
our knowledge, this is the first work which introduces the application
of a strict \fb{?} learning algorithm in the NA setting.


\fb{the section on FP is very well-read, but maybe not everything is
  relevant for the following. I'd rather try to focus on what is
  relevant for our paper and contrast approaches.  Also, it is not
  clear what is the interest of FP for multi-agent learning.  }

\paragraph{Fictitious Play}
%
Fictitious Play \cite{BrownPublished,
  BrownUnpublished}
was introduced
%by Brown in 1951
as a 'natural' way
%in which
to approximate Nash Equilibris in zero-sum games. At the time, it was
shown that a discrete variant of FP
%the algorithm would
converge to 
NE in two-player zero-sum games \cite{Robinson}. Since then, a number
of results on convergence were proved in two-player games
\cite{Miyasawa, Polak, Berger, Monderer and Sela, Monderer and
  Shapley}, including
%as well as
corresponding convergence rates \cite{Harris,
  Shapiro}.
%Many results hold for both the discrete and continuous
%variants of FP.
The relation between the discrete and continuous variants of FP was
finally solved by Hofbauer \cite{Hofbauer}, who showed that attractors
of the discrete time process are also attractors for the continuous
time variant. As such, most of the analysis on FP focuses on the
continuous time variant \cite{Ostrovski}. It was later shown in
\cite{Shapley} that there is a family of games (now called the
\emph{Shapley family}) for which FP does not converge, but
rather shows periodic behaviour. \fb{we might remove the following: This led to research which looks at
this non convergent behaviour, specifically showing the existence of
more periodic orbits as well as chaotic behaviour
\cite{Sparrow}.} However, the work looking at fictitious play
%where there are
with more than two agents is sparse. In \cite{Sela}
%considers a game in which a
multi-player games are decomposed
%by requiring that each agent engage in a
in two-player games between each pair of players in the game.
%against each of the opponent.
Each agent's payoff is given by the sum of payoffs in all of these
subgames. It was found that, if this game is zero-sum, then FP will
converge. Similar results for more than two players were found for
games with identical interests \fb{?} in \cite{MondererShapley1994}. In
\cite{FPMFG}, the action of FP was considered in a mean field game,
and it was found again that the algorithm converges for zero-sum
games. The most general result, and the one most similar to our own,
appears in
%was found by Ewerhart and Valkanova in 2020
\cite{Ewerhart}, in which
the authors show that fictitious play converges in network games, where
each agent is engaged in a two-player game with each of their
neighbours. The rate of convergence was found to be independent of the
network size.  Our work extends the analysis of FP in multi-player
games by considering its action in NA Games, so that the agents do not
play individual games against each of their neighbours but rather a
single game against the aggregate of their neighbours. In particular
we also extend a result for two player games found by Ostrovski and
van Strein \cite{Payoff Performance} which showed that FP achieves
no-regret in the multi-player setting.

\section{Preliminaries}

\fb{I'd rather not leave an empty space between section and
  subsection.}

\subsection{Network Aggregative Games}
\label{sec::NAG}

The model we consider consists of
%is that there is
a set $\agentset = \{1,
           \ldots , N \}$ of agents, who are connected through an underlying
           interaction graph.
           %This graph is given by the tuple
          %$(\agentset, (\edgeset, \weightset))$ in which $\edgeset$ is
          %the set of all pairs $(\mu, \nu) \in \agentset \times \agentset$
           %connected.
           More formally:
%        
  \begin{definition}[Interaction Graph] \label{interactiongraph}
    Given a set $\agentset$ of agents, an {\em interaction graph} $I = (\agentset, (\edgeset,
    \weightset))$ is such that
    \begin{itemize}

    \item $\edgeset \subseteq \agentset^2$.  Then, the set of
      neighbours of agent $\mu$ is denoted as $N^\mu = \{\nu \in
      \agentset \mid (\mu, \nu) \in \edgeset\}$.

    \item $\weightset \in M_N(\mathbb{R})$ is the weighted \fb{or weight?} adjacency matrix, whose elements $w^{\mu
        \nu} \in [0, 1]$ expresses the importance that agent $\mu$ places on agent $\nu$. If $(\mu, \nu) \not
        \in \edgeset$ then $w^{\mu \nu} = 0$;        $\wmunu \in (0, 1]$ otherwise.
    \end{itemize}
  \end{definition}

  Given Def.~\ref{interactiongraph} of interaction graph,
  %With the above definition,
  we introduce network aggregative (NA) games:
%
  \begin{definition}[NA Game]
    A {\em network aggregative game} is
    %defined as the
a tuple $\Gamma = (I, (\actionset{\mu},
    \utility{\mu})_{\mu \in \mathcal{N}})$, where $I$ is an
    interaction graph, and for every agent $\mu \in \mathcal{N}$, $\actionset{\mu}$ and $\utility{\mu}$
    %refer to
    are $\mu$'s set of actions (with cardinality $|\actionset{\mu}| = n$)
    and utility function respectively.
  \end{definition}

  We define the \emph{state} of agent $\mu$ to be the probability
  vector $\xmu \in \mathbb{R}^n$, where $\xmu_i$ is the probability
  with which agent $\mu$ plays action $i$.
  %It should be noted that
  This
  probability vector is often referred to as $\mu$'s \emph{mixed
    strategy}. With this in mind, we can construct, as their state
  space, the {\em unit simplex} $\Delta_\mu$ on agent $\mu$'s action set, which
  is defined
  as $\Delta_\mu := \{\xmu \in \mathbb{R}^n \, : \, \sum_i
  \xmuaction{i} = 1\}$. Also associated with each agent is a utility
  function

  The utility, for each agent $\mu$ and action profile
  $(\xmu, \xnotmu)$, is given as $u^\mu(\xmu, \xnotmu)$ in which we
  use the standard notation $-\mu$ to refer to all agents other than
  $\mu$. Notice that this requires that each agent plays the same
  strategy against all of their neighbours. What is unique about 
  NA games is the structure of the payoffs themselves. In this format,
  each agent $\mu$ receives a reference $\sigma^{\mu} = \sum_{\nu \in N^\mu} \wmunu \xnu$, which is a weighted sum
  of each of their neighbours state.
  %Formally
  %
%  \begin{equation}
%    \sigma^\mu = \sum_{\nu \in N^\mu} \wmunu \xnu.
%  \end{equation}

  Then, the agent must optimise their payoff with respect to the
  reference vector $\sigma^{\mu}$. Thus, instead of considering
  the actions of the entire population, or play individual games
  against each of their neighbours, the agent only considers
  $\sigma^\mu$ as a `measurement' of the local aggregate state and
  optimises with respect to this measurement.  This allows us to make
  the reduction $u^\mu(\xmu, \xnotmu) = u^\mu(\xmu, \refmu)$. In
  particular, we consider that the agent is engaged in a matrix game
  against the reference vector so that
  %
  \begin{equation}
    u^\mu(\xmu, \refmu) = \xmu \cdot A^\mu \refmu = \xmu
                \cdot A^\mu \weightedsum.
  \end{equation}
%
  where $A^\mu$ is the payoff matrix associated with agent $\mu$. In particular,  this means we can write the game $\Gamma$ with the payoff
  matrices $A^\mu$ in place of the utility functions
  $\utility{\mu}$. Hence, the NA game allows for the reduction of a
  multi-player game into a series of two-player games. The agent's goal
  is to maximise their payoff with respect to the reference vector. As
  such, we define the best response correspondence $BR^\mu$, which maps
  every $\refmu$ to the set $\arg \max_{y \in \Delta_\mu} {u^\mu(y,
    \refmu)}$.

  Finally, a central concept of game theory is that of the Nash
  Equilibrium, in which no rational agent has the incentive to deviate
  from their current state. This can be formalised by saying that all
  agents are playing their best response to each other.
%
  This leads naturally to the definition of a Nash Equilibrium in an NA game as
%
  \begin{definition}(NE) \label{def::NE}
    The set of vectors $\{ \NE{\mu}\}_{\mu \in \agentset}$ is a {\em
      Nash equilibrium} if, for all $\mu$,
    \begin{equation*}
    \NE{\mu} \in BR^\mu (\refmu) = \arg \max_{x \in \Delta_\mu} u^\mu(x, w^{\mu \mu} x + \sum_{\nu \in N^\mu \backslash \{\bar{\mu}\}} \wmunu \NE{\nu}).
    \end{equation*} 
  \end{definition}
    
  \begin{remark}
    %It can be seen that
The notion of Nash equilibrium in the NA game is a natural
extension of the NE in a bimatrix game.
%$(A, B)$.
In particular, if we consider an NA game with only two players and no
self-loops then the above definition yields that $ \NE{1}$ is an NE iff
%
%    \begin{equation*}
 $     \NE{1} \in BR^1 (\sigma^1) = \arg \max_{x in \Delta_1} u^1 (x, \NE{2})$,
%    \end{equation*}
%
    and similarly for $\NE{2}$. This is precisely the definition of an NE in a two player game \cite{}.
  \end{remark}

  We will show that the NE exists for an NA game in Section \ref{sec::ExistenceofNE}. 

  Finally, we note that an NA game is \emph{zero-sum} if the utilities of each agent sum to zero for any strategy set $\{ \xmu \}_{\mu \in \agentset}$. Formally,
%
%  \begin{equation}
    $\sum_\mu u^\mu(\xmu, \weightedsum) = 0$.
%  \end{equation}

\subsection{Continuous Time Fictitious Play}
\label{sec::CTFP}

  Fictitious Play requires that, at the current time, each agent
  considers the average state of their opponent in the past, and
  respond optimally (i.e., play a best response) to this state. In the
  case of an NA game, each agent considers their reference vector to
  be their opponent. As such, each agent $\mu$ must update their state
  according to the time-average of $\refmu$. To formalise this we
  write
  
  \fb{What is $\avgref{\mu}$?}

  \begin{equation}
    \avgref{\mu} = \frac{1}{t} \int_0^t \refmu(s) \; ds.
  \end{equation}

  Using this, we follow \cite{Ewerhart,Harris}
  %in the footsteps of Ewerhart \cite{} and Harris \cite{}
  to define
  Fictitious Play in continuous time, but adapted
  %with a slight adaptation for the
  to NA games.
  %
  \fb{is there any way we can make the following definition more intuitive?}
  \begin{definition}[NA-CTFP]
    A {\em network aggregative - continuous time fictitious play} is
    defined as a measurable map $m$ with components $m^\mu$ such that
    for all $\mu$ and all $t \geq 1$, $m^\mu: [0, \infty) \rightarrow
      \Delta_\mu$ satisfies $m^\mu(t) \in BR^\mu(\alpha_{\sigma}^\mu)$
      for almost all $t \geq 1$.
  \end{definition}

  We can think of this definition as saying that the player plays some arbitrary strategy before
  $t = 1$, but beyond this it must play a best response to the time average of its reference
  signal.
  
  \begin{remark}
    As an illustration, consider the NA game with two players, in which $\edgeset = \{(1, 2),
    (2, 1)\}$ and $\weightset$ is a 2x2 matrix with zeros on its leading diagonal and ones on
    the off diagonal. We write the time-average of both agents' state as
%  
%    \begin{align}
 $     \alpha^\mu(t; x) = \frac{1}{t} \int_0^{\fb{t}} \xmu(t) \, dt$, for $\mu \in \{1, 2\}$.
  %   \alpha^2 = \frac{1}{t} \int_0^T x^2(t) \, dt \\
    %    \end{align}
    
   Hence, $\alpha^\mu(t; m)$ denotes the time average of the
   strategies played by agent $\mu$ up to time $t$ when the strategies
   are given by $\xmu(t)$. Note that we often reduce the notation to
   $\alpha^\mu(t)$. Then, fictitious play requires that the agents
   update their strategy as $x^1(t) \in BR^1(\alpha^2(t))$ and $x^2(t)
   \in BR^2(\alpha^1(t))$. It can be seen, therefore, that the NA-CTFP
   is a natural extension of CTFP in the classical two-player setting
   \cite{}.
  \end{remark}

  \fb{these remarks are very helpful.}

\subsection{Assumptions}

  With the above preliminaries in place, we can state the assumptions that we make in this study.

  \begin{assumption}\label{ass::rowstochastic}
    The weighted adjacency matrix $\weightset$ is constant and
    \emph{row stochastic}, meaning that the sum elements in each row
    of $\weightset$ is equal to one. This assumption is made to ensure
    that the analysis of NA games can be derived as a natural
    extension to the classical setting of two-player games. We can
    think of the row stochastic condition as the ability of each agent
    to prioritise the state information it receives from each of its
    neighbours. It is also a classical assumption made in the analysis
    of network games \cite{one of the network game papers} and is
    straightforward to implement \cite{Eyad}.
  \end{assumption}

  \begin{assumption}\label{ass::matrixgame}
    The payoffs are given through matrix games and, therefore, are
    bilinear. Payoff matrices have a rich history in game theory and
    has generated a number of prototypical examples for economic study
    including the Prisoner's Dilemma game (see \cite{Axelrod} for an
    interesting implementation of this). They also allow for the
    design of multi-agent systems in computational settings,
    particularly in the case of task and resource allocation \cite{AGT
      and some of the Applied Game Theory Papers}. It should be noted,
    however, that game theoretic analysis is starting to consider
    various other forms of utility functions, including monotone
    \cite{Maryam} and convex \cite{Parise}. We believe that the
    analysis of Fictitious Play should follow in these developments
    and we consider it as an important area of future work.
  \end{assumption}

  \begin{assumption}\label{ass::sameactions}
    The cardinality of each action set $|\actionset{\mu}|$ is equal
    for all agents. This is a classical assumption which is made in
    most game-theoretic settings and includes, as a special case, the
    analysis of homogeneous agents \cite{}. However, it should be
    noted that, in \cite{Ewerhart}, CTFP was analysed without this
    requirement.
  \end{assumption}

  \begin{assumption}\label{ass::zerosum}
    The NA game is zero-sum in the sense that $\sum_{\mu} u^\mu(\xmu,
    \weightedsum)$ for any set of states $(x^\mu)_{\mu \in
      N^\mu}$. This is, perhaps, one of the strongest assumptions in
    our analysis, and is required for the fixed point
    analysis \fb{repetition}. Nonetheless, in Section \ref{sec::CCEConvergence}, we
    perform a regret analysis that considers the long term behaviour
    of NA-CTFP without this assumption.
  \end{assumption}

\section{Convergence of Fictitious Play}

  \subsection{Existence of the Nash Equilibrium}
  \label{sec::ExistenceofNE}
  
  As a reminder, the NE condition (Def. \ref{def::NE}) states
%
  \begin{align}
    \NE{\mu} &\in \arg\max_{x \in \Delta_\mu} u^\mu(x, w^{\mu \mu}x + \sum_{\nu \in N^\mu} w^{\mu \nu} \NE{\nu}) \nonumber \\
    & =: \arg\max_{x \in \Delta_i} \bar{u}^\mu(x, \sum_{\nu \in N^\mu} w^{\mu \nu} \NE{\nu})
  \end{align}

  where we can find $\bar{u}_i$ through the following argument
  
  \begin{align}
    u^\mu(x, w^{\mu \mu} x + \sum_{\nu \in N^\nu} w_{\mu \nu} \NE{\nu}) & = x \cdot A^\mu (w^{\mu \mu} x + \sum_{\nu \in N^\mu} w_{\mu \nu} \NE{\nu}) \\
     & = x \cdot (w^{\mu \mu} A^\mu)  x + \sum_{\nu \in N^\mu} u^{\mu \nu}(x, \NE{\nu}) \\
     & =: \bar{u}^\mu(x, \sum_{\nu \in N^\mu} w^{\mu \nu} \NE{\nu}), \nonumber
  \end{align}
  
  where $u^{\mu \nu}(\xmu, x^\nu) = \xmu \cdot A^\mu x^\nu$. Note
  that, in order to get this formulation, we had to use the assumption
  of payoffs being bilinear so that we could separate out the term in
  the weighted sum involving $x$ from $\bar{x}^\nu$.
  
  To show existence of an NE we will need the following definitions which can be found in most texts on topology \cite{Munkres}.
  
  \begin{definition}[Convexity]
  	A set $\mathcal{X}$ is \emph{convex} if, for every $x, y \in \mathcal{X}$ and every $\lambda \in [0, 1]$ we have that $\lambda x + (1 - \lambda)y \in \mathcal{X}$.
  \end{definition}

	\begin{definition}[Convex Combinations]
		The point $\sum_{i = 1}^{n} \lambda_i x_i$ is a \emph{convex combination} of the points $x_1, \ldots, x_n$ if $\lambda_i \geq 0$ for all $i$ and $\sum_i \lambda_i = 1$.
	\end{definition}

	\begin{definition}[Unit Simplex]
		The $n$-dimensional \emph{unit simplex} is defined as the convex combination of the $n$ standard basis vectors $(e_i)_{i = 1}^n$ in $\mathbb{R}^n$ in which $e_i$ has $1$ in the $i$'th component and $0$ in all others. In particular this is given by
		
		\begin{equation}
			\Delta = \{x \in \mathbb{R}^n : x = \sum_{i = 1}^n \lambda_i e_i \text{ for some } \lambda_1, \ldots, \lambda_n \geq 0, \sum_i \lambda_i = 1\}
		\end{equation}
	\end{definition}
	
	It follows from the definitions, then, that the unit simplex is a convex set. In addition we have that the simplex is \emph{closed} and \emph{bounded} in $\mathbb{R}^n$. By the Heine-Borel Theorem \cite{Royden} it is therefore also \emph{compact}. The definitions of each of these terms can be found in any standard topology reference and we point the reader to \cite{Royden} for further details.

  \begin{definition}[Upper Semi-Continuous] 
    A compact-valued correspondence $g: A \rightarrow B$ is
    \emph{upper semi-continuous} \cite{Findforthis} at a point $a \in A$ if $g(a)$ is non-empty
    and if, for every sequence $a_n \rightarrow a$ and every sequence
    $(b_n)$ such that $b_n \in g(a_n)$ for all $n$, there exists a
    convergent subsequence of $(b_n)$ whose limit point $b$ is in
    $g(a)$.
  \end{definition}

  \fb{perhaps we have to introduce some terminology from topology to keep the results self-contained?}

  \begin{theorem}[Kakutani] 
    Let $A$ be a compact and convex in a Euclidean space and $g: A \rightarrow \mathbf{P}(A)$ (where
    $\mathbf{P}(A)$ denotes the power set of A) be upper semi-continuous, with nonempty, convex and compact values.
    Then $g$ has a fixed point, that is a point $\bar{a}$ such that $\bar{a} \in g(\bar{a})$.
    \cite{Kakutani}
  \end{theorem}

  Note that, when acting on a simplex, the function $g: \Delta \rightarrow \textbf{P}(\Delta)$, where $\textbf{P}(\Delta)$ denotes all nonempty, closed and convex subsets of $\Delta$ only has to satisfy the upper-semi continuity condition to admit a fixed point.

	\ah{Probably need a quick intro into why the argmax is closed, convex and nonempty.}

  \begin{theorem}[Existence of NE]
    Under the assumption (II), namely that the payoff function achieves a bilinear property, a
    Nash Equilibrium $\{\bar{x}^\mu\}_{\mu \in \agentset}$ exists.
  \end{theorem}

\subsection{Existence of NA-CTFP}

  \begin{theorem}
    There exists a path $m$ which satisfies the property that, for all agents $\mu$, $m^\mu \in
    BR^\mu(\alpha_\sigma^\mu)$ for almost all $t \geq 1$.
  \end{theorem}

\subsection{Convergence of NA-CTFP}

With the existence of the NA-CTFP in place, we can show that it converges
to a fixed point. In particular, let $\Omega(\alpha)$ be the set of
all limit points for $\alpha(t)$. Then, a NA-CTFP path is said to have
converged if $\Omega(\alpha)$ is contained within the set of Nash
Equilibria of the game. If this is for any such NA-CTFP path, then the
game is said to have the \emph{CTFP property}. We adapt the techniques
of (Ewerhart 2020) to prove that zero-sum NA games have the CTFP
property.

  \begin{theorem}
    Any zero-sum NA game has the property that, for any NA-CTFP path $m$, the corresponding $\alpha(t; m)$ converges to a set of fixed points.
  \end{theorem}


  We now point out that, if we choose $w^{\mu \mu}$ to be zero for all $\mu$, then the final inequality yields that, for all $\mu$

  \begin{equation}
    u^\mu(y, \sum_{\nu \in N^\mu} \wmunu \xnu_\infty) \leq u^\mu(\xmu_\infty, \sum_{\nu \in N^\mu} \wmunu \xnu_\infty)
  \end{equation}

  which is precisely the Nash Equilibrium condition in the NA game. This leads to the next result

  \begin{corollary}
    With the additional assumption that, for all agents $\mu$, all zero-sum NA games have the CTFP property
  \end{corollary}

\section{NA-CTFP achieves no regret}
  \label{sec::CCEConvergence}

  In this section we aim to find some convergence structure for the
  case in which the NA game is not necessarily zero-sum. In
  particular, we show that the NA-CTFP process converges to the set of
  coarse correlated equilibria \cite{}.
  %
  \begin{definition}
    A distribution $\mathcal{D}$ over the joint action set $S =
    \times_\mu S^\mu$ is called a \emph{coarse correlated equilibrium}
    (CCE) if, for all $\mu$
    \begin{equation}
      \mathbb{E}_{s \sim \mathcal{D}}[u^\mu (s^\mu, s^{- \mu})] \geq
      \mathbb{E}_{x \sim \mathcal{D}}[u^\mu (j, s^{-
          \mu})] \hspace{1cm} \forall j \in S^\mu.
    \end{equation}
  \end{definition}

  In words, the above definition says that, if the agents are given a
  probability distribution with which they can play their actions, then
  the expected payoff, for all agents is greater than or equal to the
  payoff that they would get by playing any of their other available
  actions, assuming that the other agents keep to the distribution.

  \fb{perhaps what is below could go into a theorem/lemma.}
  
  For an NA game, if a set of actions $s = (s^1, \ldots, s^N)$ \fb{$s$ is normally used for actions} is
  drawn from a joint probability distribution $\mathcal{D}$, then
  there is a corresponding set of reference vectors $\sigma =
  (\sigma^1, \ldots, \sigma^N)$, where $\sigma^\mu = \sum_{\nu \in
    N^\mu} \wmunu s^\nu$.  Therefore, $\mathcal{D}$ is a probability
  distribution over all actions and all references.  Consider, then, a
  vector of joint distributions over actions and references which is
  generated by NA-CTFP. In particular, if by playing with NA-CTFP, the
  agents reach the state $(\xmu)_{i = 1}^N$ with references
  $(\refmu)_{i = 1}^N$, this is the vector $\mathcal{D} =
  (\mathcal{D}^1, ..., \mathcal{D}^N)$ such that
  $(\mathcal{D}^\mu)_{ij} = \xmu_i \refmu_j$. Then, the expected
  payoff that the agent would receive for playing this strategy is

  \begin{align}
    u^\mu(\xmu, \refmu) & = \xmu \cdot A^\mu \refmu \nonumber \\
    & = \sum_{i, j} (A^\mu)_{ij} \xmu_i \refmu_j \nonumber 
  \end{align}

  As such, we would say that NA-CTFP has converged to the set of CCE if, in the limit of $t
  \rightarrow \infty$, we have that for all $\mu$

  \begin{equation}
    u^\mu (\xmu, \refmu) \geq u^\mu(i', \refmu) \hspace{1cm} \forall i' \in S^\mu.
  \end{equation}

  \begin{remark}
    As usual, the notion of CCE in an NA game is a natural extension of the CCE for two player
    games. In fact, if we consider the NA game to be a two player game with no self-loops, then
    we recover exactly the definition of the CCE set in two player games \cite{PayoffPerformance}.
  \end{remark}

  \begin{remark}
    The notion of the CCE set is related to the idea of \emph{average external regret} (for the
    sake of brevity we drop the term `external' and refer the reader to \cite{AGT}). Here, we
    will present what is meant by average regret and state that if at some time $T$ all agents'
    average regret is non-positive, then the game is said to have reached the CCE set. The
    reader should consult \cite{PayoffPerformance} for an excellent exposition regarding the
    link between the CCE set and average regret in two player games which, of course, extends
    naturally to the NA game.
    
    Average regret, for agent $\mu$ is defined as

    \begin{equation}
      R^{\mu} = \max_{i' \in S^\mu} \Big\{ \frac{1}{T} \int_{0}^{T} u^{\mu}(\pure{\mu}{i'}, \sigma(t)) - u^{\mu}(m^\mu(t), \sigma(t)) \, dt \Big\},
    \end{equation}
  
    in which $\pure{\mu}{i}$ denotes the probability vector in $\Delta_\mu$ with $1$ in the slot
    $i'$ and $0$ everywhere else. Note, this is the \emph{average regret} for the agent $\mu$
    and, of course, can be related to the \emph{cumulative regret} which is used for analysis
    in \cite{Leonardos and Piliouras, Cesa-Bianchi}.  To illustrate the average regret, let us
    consider the case where each agent has only two actions. Then $u^{\mu}(x^\mu(t), \sigma(t))$
    is given by
    
    \begin{equation}
      u^{\mu}(x^\mu(t), \sigma(t)) = \sum_{ij} a_{ij} x_i^\mu \sigma_j^\mu = a_{11} x_1^\mu \sigma_1^\mu + a_{12} x_1^\mu \sigma_2^\mu + a_{21} x_2^\mu \sigma_1^\mu + a_{22} x_2^\mu \sigma_2^\mu
    \end{equation}
  
    On the other hand, let us consider that agent $\mu$'s first strategy maximises $u^{\mu}(\pure{\mu}{1}, \sigma(t))$, then
  
    \begin{equation}
      u^{\mu}(\pure{\mu}{1}, \sigma(t)) = \sum_{ij} a_{1j} x_i^\mu \sigma_j^\mu = a_{11} x_1^\mu \sigma_1^\mu + a_{12} x_1^\mu \sigma_2^\mu + a_{11} x_2^\mu \sigma_1^\mu + a_{12} x_2^\mu \sigma_2^\mu 
    \end{equation}
  
    By comparing the two expanded expressions, we can see that the latter gives the reward that agent $\mu$ would have received had they played action $1$ throughout the entire play, assuming that the behaviour of the other agents (encoded in $\sigma$) does not change. As such, this is a measure of agent $\mu$'s regret, in hindsight, for not playing action $1$ the entire time. An agent achieves \emph{no regret} if $R^\mu$ is non-positive.
  \end{remark}



  \begin{theorem}
    Assuming that $w^{\mu \mu} = 0$, then for any choice of payoff matrix, agents following the
    NA-CTFP process achieve \emph{no regret} in the limit $t \rightarrow \infty$, i.e.
    \begin{equation}
      \lim_{T \rightarrow \infty} \max_{x_{i'}^\mu \in S^\mu} \Big\{ \frac{1}{T} \int_{0}^{T} u^{\mu}(x_{i'}^\mu(t), \sigma(t)) - u^{\mu}(m^\mu(t), \sigma(t)) \, dt \Big\} = 0
    \end{equation}

    In particular, NA-CTFP converges to the set of CCE.

  \end{theorem}
  

\section{Numerical Experiments}

  \subsection{Non-convergent examples} \label{sec::NonConv}
  The purpose of this section is to show that, whilst wwe have shown that it converges in zero-sum games, NA-CTFP is not guaranteed to converge in general games, and can in fact give rise to a rich variety of dynamics.

  
  
  As a example of nonconvergence we consider the Shapley family of games \cite{}. In a two player bimatrix game $(A_\beta, B_\beta)$ this is given by

  \begin{equation}
    A_\beta = \begin{bmatrix}
      1 & 0 & \beta \\
      \beta & 1 & 0 \\
      0 & \beta & 1
    \end{bmatrix}, \; B_\beta =  \begin{bmatrix}
      - \beta & 1 & 0 \\
      0 & -\beta & 1 \\
      1 & 0 & -\beta
    \end{bmatrix}
  \end{equation}

  where $\beta \in (0, 1)$. In \cite{} this was shown to produce both periodic and chaotic behaviour for different choices of $\beta$. 

  \begin{figure}[t]
    \centering
    \includegraphics[width = 0.6\textwidth]{Figures/ThreePlayerNetwork.png}
    \caption{\label{fig::ThreePlayerNetwork} A NA game where the network is defined by a chain of
    three players, as described in Section \ref{sec::NonConv}.}
  \end{figure}

  As an adaptation, we take the example of a three player chain, as depicted in Figure \ref{fig::ThreePlayerNetwork}. In this example, we first assume that the network is simple (i.e. there are no self-loops and $w^{\mu \mu} = 0$). The aggregation matrix can be given as

  \begin{equation}
    W = \begin{bmatrix}
      0 & 1 & 0 \\
      w & 0 & 1 - w \\
      0 & 1 & 0
    \end{bmatrix}, \; w \in (0, 1).
  \end{equation}


  We first consider the zero-sum case to show that it does indeed converge to an equilibrium as expected. Note that the zero-sum condition given for the three player chain is given as

  \begin{equation}
    x \cdot A y + y \cdot B (w x + (1-w)z) + z \cdot C y = 0. \hspace{0.5cm} \forall x, y, z \in \Delta_1 \times \Delta_2 \times \Delta_3
  \end{equation}

  in which we use the notation that $x, y, z$ (resp. $A, B, C$) denote the strategies (resp. payoffs) of agents 1, 2 and 3 respectively. This condition is satisfied if we fix $B$ and choose

  \begin{align} \label{eq::zeroSumShapley}
    A & = - w B^T \nonumber \\
    C & = - (1 - w) B^T. 
  \end{align}

  As such in the following example, we will set $B = B_\beta$ with the choice $\beta \approx 0.576$ and set $A$ and $C$ according to the above with the choice $w \approx 0.288$. The orbits that these payoff matrices generate can be seen in Figure \ref{fig::convergentShapley}, in which, for each player, they converge to the Nash Equilibrium which, for each player, lies in the centre of the simplex.

  Let us now make the slight modification in the definition of C so that

  \begin{equation} \label{eq::nonzeroSumShapley}
    C  = - (1 - w) B, 
  \end{equation}

  with no alteration to $A$. The modification itself is small, however it results in the zero-sum
  assumption being violated. With the same choices of $\beta$ and $w$, this results in the periodic
  orbit seen in Figures \ref{fig::nonconvergentShapley} and \ref{fig::3PlayerChainNoNoise}. Here,
  the orbits reach a stable limit cycle which to be centred around the interior NE.

  \begin{figure}[t]
    \centering
    \begin{subfigure}[b]{0.45 \textwidth}
      \includegraphics[width = \textwidth]{Figures/convergentShapley.png}
    \caption{\label{fig::convergentShapley}}
    \end{subfigure}
    \begin{subfigure}[b]{0.45 \textwidth}
      \includegraphics[width = \textwidth]{Figures/nonConvergentShapley.png}
      \caption{\label{fig::nonconvergentShapley}}
    \end{subfigure}
    \caption{\label{fig::Shapley} Orbits of the Fictitious Play in the Three Player Chain (c.f.
    Figure \ref{fig::ThreePlayerNetwork}) with payoff matrices given by (a)
    (\ref{eq::zeroSumShapley}) showing convergence to the interior NE (b)
    (\ref{eq::nonzeroSumShapley}) showing cycles around the interior NE.}
  \end{figure}


  \begin{figure}[t]
    \centering
    \includegraphics[width = 0.7 \textwidth]{Figures/3PlayerChainNoNoise.png}
    \caption{\label{fig::3PlayerChainNoNoise} Distance to the Nash Equilibrium strategy along orbits
    of FP showing oscillations which are bounded away from the NE (i.e. they do not cross the NE).
    The x-axis shows logarithmic time since FP includes a factor of $1/t$, which has the effect of
    slowing down the orbit over time.}
  \end{figure}

  As such, we can see that convergent behaviour is not necessarily the norm in the NA-CTFP dynamics. In fact, for the family of games discussed above, we were unable to find non-periodic behaviour for any choice of $\beta$ strictly between 0.5 and 1 for any $w$ between 0.2 and 0.8 (so that the influence of player 1 and player 3 on player 2 is not negligible). This suggests that, far from being rare, in fact NA-CTFP lends itself to an incredibly rich variety of dynamics which can be explored as future work.
  

  \subsection{Addition of Noise}

  \begin{figure}[t]
    \centering
    \includegraphics[width = \columnwidth]{Figures/Noise10Player.png}
    \caption{\label{fig::Noise10Player} Trajectories of NA-CTFP in a 10 player game with additive noise (Left) No noise is introduced and learning converges directly to an NE. (Middle) $\gamma = 0.1$, the trajectories converge to a fixed point but removed from the NE. (Right) $\gamma = 0.5$, the trajectories converge to a fixed point which is even further away from the NE.}
  \end{figure}

  \begin{figure}[t]
    \centering
    \includegraphics[width = \columnwidth]{Figures/Noise20Player.png}
    \caption{\label{fig::Noise20Player} Trajectories of NA-CTFP in a 20 player game with additive noise (Left) No noise is introduced and learning converges directly to an NE. (Middle) $\gamma = 0.05$, the trajectories converge to a fixed point but removed from the NE. (Right) $\gamma = 0.1$, the trajectories converge to a fixed point which is even further away from the NE.}
  \end{figure}

  \begin{figure}[t]
    \centering
    \includegraphics[width = \columnwidth]{Figures/3PlayerChainNoise.png}
    \caption{\label{fig::3PlayerChainNoise} Trajectories of NA-CTFP on the Three Player Chain of Section \ref{sec::NonConv} with additive noise. (Left) $\gamma = 0.1$ leads to a decrease in the size of the cyclic orbit (Middle) $\gamma = 0.5$, no periodicity is seen but the trajectory converges to the NE (Right) $\gamma = 0.75$, NA-CTFP still converges, though after a greater amount of time has elapsed.}
  \end{figure}

  The fictitious play process in NA games requires that, at each time step, an agent takes a
  `measurement' of the aggregate strategy of its neighbours. It is on this measurement that they
  update their own strategy. It stands to reason then, that in real environments this measurement
  may be corrupted by noise. 
  
  As such, we investigate the effect that introducing additive noise has
  on NA-CTFP in a zero-sum NA game. We do this in the following manner: at each time step, the
  reference signal $\refmu(t)$ is adjusted to $\refmu + \gamma \xi$ where $\xi$ is drawn from the
  standard normal distribution (zero mean and unit variance). By varying $\gamma$, we vary the
  strength of the noise. We vary $\gamma$ up to $0.5$ since, above this value, noisy measurements
  are likely to lie outside of the simplex. Since $\refmu$ is constrained to lie within $\Delta$, we
  can consider the range $\gamma \in [0, 0.5]$ to be the \emph{physical region}, in which noise is meaningful.

  In Figure \ref{fig::Noise10Player}, we consider a zero-sum NA game with 10 players. When there is
  no noise, it can be seen that FP reaches a fixed point which, since we set $w^{\mu \mu} = 0$,
  corresponds to an NE. After increasing $\gamma$, however, we find that the agents no longer
  converge to this NE, but rather shift away from it. What is interesting, however, is that the
  orbits do still reach a stationary state in the long run which suggests that FP is still able to
  converge with the introduction of noise. Figure \ref{fig::Noise20Player} shows similar behaviour
  for $N = 20$, and in our experiments we found this to be ubiquitous regardless of the choice of
  $N$ or $n$. 

  In Figure \ref{fig::3PlayerChainNoise} we revisit the Three Player Chain of Section
  \ref{sec::NonConv}, now under the influence of additive noise. For the sake of brevity, we only
  display the distance to the Nash Equilibrium of the first player's action, since the other agents
  behave in the same way. It can be seen that a small amount of noise has the effect of decreasing
  the size of the periodic orbit (c.f. Figure \ref{fig::3PlayerChainNoNoise}). However, as $\gamma$
  is increased to $0.5$, the algorithm seems to exhibit convergence to the NE. The implication is
  that the addition of noise may cause periodic behaviour to break and lead to the Nash Equilibrium.
  An interesting point to note is that this behaviour is in stark contrast to the replicator
  dynamic (RD) \cite{Maynard-Smith}, another adaptive algorithm linked to multi agent learning
  \cite{CyclesAdversarialLearning}. In \cite{Imhof} and \cite{Galla}, it was found that the
  introduction of random mutations can remove convergent behaviour and instead lead to periodicity. 
  
\section{Concluding remarks}
	In this work, we have considered the action of the Fictitious Play learning algorithm in Network Aggregative Games and investigated its long term behaviour through a continuous time analysis. We find that, under a zero-sum condition, the algorithm converges to a fixed point. However, we find experimentally that this is not always the case. In fact, we find a family of NA games, based on the Shapley family, for which FP cycles about the NE. For these cases, we also perform a regret analysis which shows that, regardless of the type of game, the FP algorithm achieves no regret. 
	
	We also investigate the influence of noise on the algorithm and find that even with the introduction of additive noise, FP converges to a fixed point, though not necessarily the NE. In fact, for our cyclic family of games, we find that the introduction of noise can actually remove the periodicity, resulting in FP converging to a fixed point.
	
	Our work opens a number of lines for future work. Most notable is the effect of noise. It would be prudent to analyse this theoretically, as was done in \cite{MFGFP}, and consider the conditions under which FP will still converge to a fixed point. Furthermore, it would be interesting to investigate the phenomenon we report experimentally in a theoretical framework. Namely, the question of why noise breaks periodicity in FP and results in convergence to an NE should be investigated and, indeed, this is a line which we are currently pursuing.
	
	In addition, we note that the \emph{Mann Iteration}, a method of approximating fixed points which is investigated in \cite{Parise2020}, shares a remarkably similar structure to the discrete variant of FP. This may present an avenue by which NA-CTFP may be analysed in the case of convex cost functions. 
	
	Finally, we note that in recent years FP in two player games has shown a remarkable variety of dynamical behaviours, including periodicity and chaos. In our work we have shown convergence to a fixed point and, through experiments, periodicity. It stands to reason, therefore, that a greater variety of dynamical behaviours exist for NA-CTFP for certain classes of games. It would be important to determine what these classes are. Short from being merely a curiosity, this would allow for the identification of games in which NA-CTFP leads to inherently unpredictable behaviour, an important question from the point of view of building Safe and Trusted AI.

\section*{Broader Impact}

The dynamics of learning is an important consideration for all practitioners. In particular, it has
been shown a number of times \cite{noneqmbehaviour} that convergence of learning cannot always be
assumed. Rather, learning generally presents much more complex dynamics \cite{gallaandfarmer}, which
only increases as the number of players increases \cite{sanders}. Our work presents practitioners
who applies Fictitious Play with a case in which rigorous stability properties may be guaranteed.
We also elucidate the behaviour of the algorithm under more general assumptions, both by
understanding its regret properties as well as through an experimental study of the impact of noise.

As regards FP  itself, the learning strategy has strong applications in robotic control
\cite{Smyrnakis, Hernandez, Sharma} as well as economic modelling \cite{vonNeumann}. As such we
Furthermore, the algorithm has links to other learning protocols including the replicator dynamic
\cite{Benaim} and reinforcement learning \cite{LeslieandCollins}. Therefore, we believe that an
understanding of FP has subsequent impacts on a number of fields.

Finally, our work has a strong impact on the study of the Network Aggregative Game, which has strong
applications in multi agent control \cite{GrammaticoAggregativeControl}. We 
believe that our work makes a strong step towards ensuring that systems which learn and adapt on
NA games maintain stability and, therefore, can be considered safe.

\begin{ack}
Use unnumbered first level headings for the acknowledgments. All acknowledgments
go at the end of the paper before the list of references. Moreover, you are required to declare 
funding (financial activities supporting the submitted work) and competing interests (related financial activities outside the submitted work). 
More information about this disclosure can be found at: \url{https://neurips.cc/Conferences/2020/PaperInformation/FundingDisclosure}.


Do {\bf not} include this section in the anonymized submission, only in the final paper. You can use the \texttt{ack} environment provided in the style file to autmoatically hide this section in the anonymized submission.
\end{ack}

\section*{References}

References follow the acknowledgments. Use unnumbered first-level heading for
the references. Any choice of citation style is acceptable as long as you are
consistent. It is permissible to reduce the font size to \verb+small+ (9 point)
when listing the references.
{\bf Note that the Reference section does not count towards the eight pages of content that are allowed.}
\medskip

\small

[1] Alexander, J.A.\ \& Mozer, M.C.\ (1995) Template-based algorithms for
connectionist rule extraction. In G.\ Tesauro, D.S.\ Touretzky and T.K.\ Leen
(eds.), {\it Advances in Neural Information Processing Systems 7},
pp.\ 609--616. Cambridge, MA: MIT Press.

[2] Bower, J.M.\ \& Beeman, D.\ (1995) {\it The Book of GENESIS: Exploring
  Realistic Neural Models with the GEneral NEural SImulation System.}  New York:
TELOS/Springer--Verlag.

[3] Hasselmo, M.E., Schnell, E.\ \& Barkai, E.\ (1995) Dynamics of learning and
recall at excitatory recurrent synapses and cholinergic modulation in rat
hippocampal region CA3. {\it Journal of Neuroscience} {\bf 15}(7):5249-5262.

\end{document}
